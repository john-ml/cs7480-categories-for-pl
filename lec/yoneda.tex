\chapter{Yoneda I}

\noindent In set theory, functions enjoy the following nice properties.
\begin{enumerate}
\item Extensional equality: two functions \(f,g : X \to Y\)
  are equal if and only if \(f(x) = g(x)\) for all elements \(x\) of \(X\).
\item Elementwise definition:
  to define a function \(f : X \to Y\),
  it suffices give a relation,
  called the \emph{graph of \(f\)},
  such that each element \(X\)
  is related exactly one element of \(Y\).
\end{enumerate}
The presence of elements makes functions much easier to work with.
For instance, compare the proof \(X \times (Y \times Z) \cong (X\times Y) \times Z\)
as sets with the abstract proof for an arbitrary category with products.

In this chapter we will see how to recover a similar kind of element-wise reasoning
style that works in an arbitrary category.
While objects of categories don't have elements,
they do have \emph{generalized elements},
which bring much of the benefits of elements in set theory
to working in arbitrary categories.

The starting point for this generalization is the following
proposition, which recasts ``element of a (finite) set'' into categorical language.
\begin{proposition}[Elements in \(\FinSet\)]
  Let \(X\) be an object of \(\FinSet\).
  Elements of \(X\) are in bijection
  with morphisms of \(\FinSet\)
  from \(1\) to \(X\).
\end{proposition}
In light of this proposition, properties (1) and (2)
above translate into the following statements about \(\FinSet\):
\begin{itemize}
\item Extensional equality:
  Two morphisms \(f,g : X \to Y\)
  in \(\FinSet\) are equal if and only if \(f \circ x = g \circ x\)
  for all morphisms \(x : 1 \to X\).
\item Elementwise definition:
  to define a morphism \(f : X \to Y\)
  of \(\FinSet\),
  it suffices to give a relation \(R\)
  such that each
  morphism \(1 \to X\)
  is related to exactly one morphism \(1 \to Y\).
\end{itemize}


\section{From sets to categories}

How to generalize the above situation to an arbitrary category?
An arbitrary category might not have a terminal object \(1\).
Even if they do, it's not guaranteed that functions can be
defined and tested for equality simply via morphisms \(1 \to X\).

The trick is to replace the special object \(1\)
with a \emph{quantification over all possible objects}.
Compare the statement of extensionality for \(\FinSet\)
above with the following proposition, which holds in any category:

\begin{proposition}[Generalized extensionality] \label{prop:generalized-extensionality}
  Let \(f,g : X \to Y\) be two morphisms in a category \(\calC\).
  Suppose that \(f \circ x = g \circ x\) for all objects \(\Gamma\)
  and all morphisms \(x : \Gamma \to X\).
  Then \(f = g\).
\end{proposition}
\begin{proof}
  Letting \(\Gamma = X\) and \(x = \idt_X\) gives \(f \circ \idt_X = g \circ \idt_X\).
  Simplifying yields \(f = g\).
\end{proof}

\begin{definition}[Generalized element]
  Let \(X\) and \(\Gamma\) be objects of a category \(\calC\).
  A \emph{generalized element of \(X\) at stage \(\Gamma\)}
  is a morphism \(x : \Gamma \to X\).
  The set of generalized elements of \(X\) at stage \(\Gamma\)
  will be written \(\genelt{X}{\Gamma}\).
\end{definition}

In this new language, Proposition~\ref{prop:generalized-extensionality}
says that two morphisms of a category are equal if
they act the same on all generalized elements.

Generalizing the principle of elementwise definition
is trickier. Since the single object \(1\) has been replaced by a quantification over
all objects \(\Gamma\), the notion of ``function'' as graph
needs to be adjusted similarly.

\begin{definition}[Generalized function]
  Let \(X,Y\) be objects of a category \(\calC\).
  A \emph{generalized function} \(\varphi\) from \(X\) to \(Y\)
  is a family of functions
  \(\varphi_\Gamma \subseteq \genelt{X}{\Gamma} \to \genelt{Y}{\Gamma}\)
  indexed by objects \(\Gamma\) of \(\calC\)
  that respects ``change of stage''
  in the sense that for all \(p : \Gamma' \to \Gamma\)
  and all \(x \in \genelt{X}{\Gamma}\)
  it holds that \(\varphi_\Gamma(x\circ p) = \varphi_\gamma(x)\circ p\).
\end{definition}

\begin{proposition}[Generalized elementwise definition]
  \label{prop:generalized-elementwise-definition}
  Let \(X,Y\) be objects of a category \(\calC\).
  Let \(\varphi\) be a generalized function from \(X\) to \(Y\).
  There exists a morphism \(f : X \to Y\)
  such that \(\varphi_\Gamma(x) = f \circ x\)
  for all objects \(\Gamma\) and all \(x \in \genelt{X}{\Gamma}\).
\end{proposition}

\begin{proposition}
  \(X \times (Y\times Z) \cong (X \times Y) \times Z\)
  in any category \(\calC\) with products.
\end{proposition}
\begin{proof}
  Let \(\varphi\) be a family of functions
  from generalized elements of \(X \times (Y\times Z)\)
  to generalized elements of \((X\times Y) \times Z\)
  defined as follows:
  \[
  \varphi_\Gamma\angled{x,\angled{y,z}}
  =\angled{\angled{x,y},z}
  \text{ for all \(x:\Gamma\to X,y:\Gamma\to Y,z:\Gamma\to Z\)}
  \]
  This function respects change of stage:
  \begin{align}
    \varphi_\Gamma\angled{x,\angled{y,z}}\circ p
    &= \angled{\angled{x,y},z}\circ p\\
    &= \angled{\angled{x\circ p,y\circ p},z \circ p}\\
    &= \varphi_\Gamma\angled{x\circ p,\angled{y\circ p,z\circ p}}\\
    &= \varphi_\Gamma(\angled{x,\angled{y,z}}\circ p).
  \end{align}
  Hence \(\varphi\) is a generalized function from \(X \times (Y\times Z)\)
  to \((X\times Y) \times Z\) and,
  by generalized elementwise definition,
  there exists a morphism \(f : X \times (Y\times Z) \to (X\times Y) \times Z\),
  such that \(f\circ \angled{x,\angled{y,z}}
  = \varphi_\Gamma\angled{x,\angled{y,z}}
  = \angled{\angled{x,y},z}\)
  for all objects \(\Gamma\) and morphisms \(x:\Gamma\to X,y:\Gamma\to Y,z:\Gamma\to Z\).

  Similarly, we can define a morphism \(g\) going the other way,
  in terms of the following generalized function \(\psi\):
  \[
  \psi_\Gamma\angled{\angled{x,y},z} = \angled{x,\angled{y,z}}.
  \]

  Now for
  all objects \(\Gamma\) and morphisms \(x:\Gamma\to X,y:\Gamma\to Y,z:\Gamma\to Z\),
  it holds that
  \begin{align}
    (f\circ g)\circ \angled{\angled{x,y},z}
    &= f\circ (g\circ \angled{\angled{x,y},z}) \\
    &= f\circ (\psi_\Gamma\angled{\angled{x,y},z}) \\
    &= f\circ \angled{x,\angled{y,z}} \\
    &= \varphi_\Gamma\angled{x,\angled{y,z}} \\
    &= \angled{\angled{x,y},z} \\
    &= \idt_{(X\times Y) \times Z} \circ \angled{\angled{x,y},z}
  \end{align}
  so \(f\circ g = \idt_{(X\times Y) \times Z}\)
  by generalized extensionality.

  An analogous argument gives \(g \circ f = \idt_{X\times (Y\times Z)}\).
  Hence \(f\) and \(g\) form an isomorphism
  between \(X \times (Y\times Z)\) and \((X\times Y) \times Z\).
\end{proof}
Note the similarity between this proof and the standard set-theoretic
argument that \(X \times (Y\times Z) \cong (X \times Y) \times Z\)
when \(X,Y,Z\) are sets.
The key differences in this more general setting are (1)
the quantification over stages \(\Gamma\)
and (2) the checks that generalized functions respect change of stage.

If you want to learn more about this perspective, check out
Tom Leinster's ``\href{https://webhomes.maths.ed.ac.uk/~tl/elements.pdf}{Doing without diagrams}''.

\chapter{Yoneda II: indexed set theory and representability}

Throughout this section we will work with an arbitrary catetgory \(\calC\).

\begin{definition}
  A \emph{\(\calC\)-indexed set}
  consists of a family of sets \(PX\) indexed by objects \(X\) of \(\calC\),
  along with ``substitution functions'' \(Pf : PY \to PX\)
  for each morphism \(f : X \to Y\) in \(C\).
  Given an element \(y \in PY\)
  and a morphism \(f : X \to Y\),
  the element \((Pf)(y)\) will be written \(y \cdot_P f\)
  and called the ``substitution of \(y\) by \(f\)''.
  Every \(\calC\)-indexed set \(P\)
  must satisfy the following laws:
  \begin{itemize}
  \item Identity: \(x \cdot_P \idt_X = x\) for all objects \(X\) and \(x \in PX\)
  \item Composition: \(z \cdot_P (f\circ g) = z \cdot_P f \cdot_P g\)
    for all objects \(f : Y \to Z\) and \(g : X \to Y\) and \(z \in PZ\).
  \end{itemize}
\end{definition}

\begin{example}
  Recall that \(\mathsf{STLC}\) is a category
  whose objects are types and whose morphisms \(A \to B\)
  are terms \(x : A \vdash M : B\) quotiented by \(\beta\eta\)-equivalence.
  For each type \(A\),
  the following defines a \(\mathsf{STLC}\)-indexed set \(\mathsf{Tm}(A)\):
  \begin{align}
  \mathsf{Tm}(B)(A) &= \text{the set of morphisms \(A\to B\)} \\
  \mathsf{Tm}(B)(N : A'\to A)
  &= (x\ofty A \vdash M : B) \mapsto (x'\ofty A' \vdash M[N/x] : B)
  \end{align}
\end{example}
This example is a special case of a canonical kind of \(\calC\)-indexed set.
\begin{definition}
  The \emph{representable \(\calC\)-indexed set at \(X\)}
  is written \(\yo X\) and defined by
  \begin{align}
    (\yo X)(A) &= \text{the set of morphisms \(A \to X\)} \\
    (\yo X)(A)(s : A' \to A) &= (f : A \to X) \mapsto (f \circ s : A' \to X)
  \end{align}
\end{definition}

\begin{definition}
  \sloppy
  Let \(P\) and \(Q\) be \(\calC\)-indexed sets.
  A \emph{\(\calC\)-indexed function}
  from \(P\) to \(Q\),
  written \(\alpha : P \Rightarrow Q\),
  is a family of functions \(\alpha_X : PX \to QX\)
  that ``respects substitution''
  in the sense that \(\alpha_{X'}(x\cdot_P p) = \alpha_X(x)\cdot_p p\).
\end{definition}

\begin{itemize}
\item \(\calC\)-indexed functions can be composed:
  the composition of \(\alpha : Q \Rightarrow R\)
  and \(\beta: P \Rightarrow Q\),
  written \(\alpha\circ\beta : P \Rightarrow R\),
  is defined by \((\alpha\circ\beta)_X = \alpha_X \circ \beta_X\).
\item For each \(\calC\)-indexed set \(P\),
  there is a \(\calC\)-indexed function \(\idt_P : P \Rightarrow P\)
  defined by \(\idt_{P,X} = \left(PX \xrightarrow{\idt_{PX}} PX\right)\).
\end{itemize}
By now you may have noticed that \(\calC\)-indexed sets and functions
look like they ought to form a category. More on this later.

\begin{definition}
  Two \(\calC\)-indexed sets \(P,Q\)
  are \emph{isomorphic}
if there are \(\calC\)-indexed functions \(\alpha : P \Rightarrow Q\)
and \(\beta : Q \Rightarrow P\)
such that \(\alpha\circ\beta=\idt_Q\) and \(\beta\circ\alpha=\idt_P\).
\end{definition}

\begin{definition}
  A \(\calC\)-indexed set \(P\) is \emph{representable}
  if there is an isomorphism \(\alpha : P \cong \yo X : \beta\)
  for some object \(X\) of \(\calC\).
\end{definition}

\chapter{Yoneda III: universal constructions, elements, and properties}

\begin{definition}
  \sloppy
  Given two \(\calC\)-indexed sets \(P\) and \(Q\),
  their \emph{product} \(P\times Q\)
  is the \(\calC\)-indexed set defined by
  \((P\times Q)(X) = P(X)\times Q(X)\),
  with substitution defined by
  \((x,y)\cdot_{P\times Q} p = (x\cdot_P p, y\cdot_Q p)\).
\end{definition}

\begin{proposition}
  Let \(X\) and \(Y\) be two objects of a category \(\calC\).
  An object \(P\) is a product of \(X\) and \(Y\)
  if and only if \(\yo P\) is isomorphic to \(\yo X \times \yo Y\).
\end{proposition}

\begin{definition}
  Given two objects \(X\) and \(Y\) of a category \(\calC\),
  there is a \(\calC\)-indexed set of \emph{closures},
  written \(\mathsf{Clos}(X,Y)\),
  defined by \(\mathsf{Clos}(X,Y)(\Gamma) = \calC(\Gamma\times X, Y)\),
  with substitution defined by
  \[
    \mathsf{Clos}(X,Y)(s : \Gamma'\to \Gamma)(f : \Gamma\times X \to Y)
    = \angled{\gamma',x} \mapsto f \circ \angled{s\circ \gamma', x}
  \]
  on generalized elements.
\end{definition}

\begin{proposition}
  Let \(X\) and \(Y\) be two objects of a category \(\calC\).
  An object \(E\) is the exponential \(Y^X\)
  if and only if \(\yo E\) is isomorphic to \(\mathsf{Clos}(X,Y)\).
\end{proposition}

\begin{proposition}
  Suppose the \(\calC\)-indexed set \(P\) is represented
  by the object \(X\).
  Then there exists an element \(u \in PX\),
  called the \emph{\(P\)-universal element},
  which satisfies the following \emph{universal property}:
  for any object \(\Gamma\) and any element \(g \in P\Gamma\),
  there exists a unique morphism \(\hat g : \Gamma \to X\)
  such that \(g = u \cdot_P \hat g\).
\end{proposition}

\begin{itemize}
\item In the case of the product, with \(\yo P \cong \yo X \times \yo Y\),
  the universal element is an element \(u \in \calC(P,X) \times \calC(P,Y)\),
  which is a pair of morphisms \(P\to X, P\to Y\).
  The universal property of this pair is that any other pair \(\Gamma \to X,\Gamma\to Y\)
  factors uniquely through it---precisely the universal property of products
\item In the case of exponents, with \(\yo E \cong \calC(A\times(-),B)\),
  the universal element is an element \(u \in \calC(A \times E,B)\),
  which is a morphism \(A \times E \to B\).
  The universal property of this morphism is that any other morphism
  \(A \times \Gamma \to B\)
  factors uniquely through it---precisely the universal property of exponents.
\end{itemize}

\chapter{Yoneda IV}

%% - Grothendieck universes
%% - The category Set
%% - Functors; opposite categories; C-indexed sets as functors
%% - Natural transformations; C-indexed functions as natural transformations
%% - The Yoneda lemma (without showing naturality in P,c)

%% \begin{itemize}
%% \item Functors form a category where morphisms are natural transformations \begin{itemize}
%%     \item Natural transformations as homotopies between diagrams
%%       (the \(C\to D^\to\) example revisited)
%%     \item Natural transformations as polymorphic maps
%%   \end{itemize}
%% \item Definition of Yoneda embedding
%% \item Restatement of universal properties from previous week in terms of Yoneda embedding
%% (e.g., for products, \(\yo(a\times b) \cong \yo(a)\times \yo(b)\).)
%% \item Yoneda embedding full and faithful. Use this to quickly prove some basic facts:
%%   associativity and commutativity of sums and products, distributivity of products over sums,
%%   ...
%%   These imply type isomorphisms in STLC and Set, entailments of propositional logic.
%% \item Yoneda preserves limits.
%%   This gives a quick proof that all limits can be constructed from products and equalizers.
%%   From this we can compute limits in STLC (it won't have all of them
%%   but it will have some).
%% \end{itemize}

%% \todo: \begin{itemize}
%% \item Slice over c is category of elements of yo c
%% \end{itemize}
