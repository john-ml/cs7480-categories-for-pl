\chapter{Yoneda}

\begin{itemize}
\item Universal properties help justify a construction,
  but are pretty bad for making use of it.
  (For example: try to prove associativity of products using the universal property alone.)
  The Yoneda lemma and associated reasoning principles are more useful for this.
\item Functors form a category where morphisms are natural transformations \begin{itemize}
    \item Natural transformations as homotopies between diagrams
      (the \(C\to D^\to\) example revisited)
    \item Natural transformations as polymorphic maps
  \end{itemize}
\item Definition of Yoneda embedding
\item Restatement of universal properties from previous week in terms of Yoneda embedding
(e.g., for products, \(\yo(a\times b) \cong \yo(a)\times \yo(b)\).)
\item Yoneda embedding full and faithful. Use this to quickly prove some basic facts:
  associativity and commutativity of sums and products, distributivity of products over sums,
  ...
  These imply type isomorphisms in STLC and Set, entailments of propositional logic.
\item Yoneda preserves limits.
  This gives a quick proof that all limits can be constructed from products and equalizers.
  From this we can compute limits in STLC (it won't have all of them
  but it will have some).
\end{itemize}

\todo: \begin{itemize}
\item Slice over c is category of elements of yo c
\end{itemize}

