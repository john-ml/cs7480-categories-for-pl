\chapter{Adjunctions}

\begin{itemize}
  \item So far we've mostly focussed on characterizing properties 
  of an individual category. Now let's broaden our focus to discuss 
  in more detail
  relationships between categories.
  \item First, let's study the strongest possible relationship between
  two categories: does it mean for two categories to be equivalent? 
  Following intuition from set isomorphism, we will want to design 
  two functors between the categories that behave like inverses.
  \item For this definition we will need a way of talking equivalence 
  of functors:
  \begin{definition}[Natural isomorphism of functors]
    Let $\calC$ and $\calD$ be categories and $F, G : \calC \to \calD$ 
    be functors. Then, $F$ and $G$ are \emph{naturally isomorphic},
    written $F \cong G$, if there is a natural transformation $\alpha : F \Rightarrow G$
    whose components are isomorphisms.

  \end{definition}
  % todo: make this just be X x Y

  % \begin{example}
  %   Let $\calC$ be the preorder category on natural numbers, and let 
  %   $\calD$ be a category whose objects are pairs of natural numbers 
  %   and where there is a morphism $(x,y) \to (z, w)$
  %   if and only if $x \le z$ or $y \le w$.

  %   Then, let $F : \calC \to \calD$ be $F = X \mapsto (0, X)$  and 
  %   $G : \calC \to \calD$ be $G  = X \mapsto (X, 0)$. Then,
  %   these two functors are naturally isomorphic along the natural 
  %   transformation
  %   $\alpha : F \Rightarrow G$ where components $\alpha_X : F(X) \to G(X)$
  %   swap the order of tules, $\alpha_X = (0, X) \to (X, 0)$.
  % \end{example}

  \item Now, to give an equivalence of categories, we must show that there exists a pair of functors $F$ and 
  $G$ between them whose compositions are naturally isomorphic to 
  the identity functors on each category:
  \begin{definition}[Equivalence of categories]
    \sloppy
    Let $\calC$ and $\calD$ be categories. Then, $\calC$ is equivalent
    to $\calD$, written $\calC \cong \calD$, if there are two functors 
    $F : \calC \to \calD$ and $G: \calC \to \calD$ satisfying
    $F \circ G \cong \id_\calC$ and $G \circ F \cong \id_\calD$, where $\id_\calC$ 
    and $\id_\calD$ are the identity functors on $\calC$ and $\calD$ respectively.
  \end{definition}
  \item There are some surprising categorical equivalences that are worth 
  thinking about:
  \begin{proposition}
    The category of finite pointed sets $\mathsf{FinSet}^*$ is equivalent to the category 
    of finite sets and partial functions $\mathsf{FinSet}_\text{par}$. 
  \end{proposition}

  The intuition behind this equivalence is interesting. First let's identify 
  the functors that form the equivalence. Let $F :
  \mathsf{FinSet}_\text{par} \to \mathsf{FinSet}^*$. The action 
  on objects is straightforward: $F(X) = (X \uplus \{\star\}, \star)$
  i.e., $F$ sends a set $X$ to a pointed set with $X$ extended by 
  some element $\star$ with $\star$ as the point.
  The action on morphisms is interesting. Let $f : X \to Y$ be a partial 
  function. Then, we define:
  \begin{align*}
    F(f) &: (X \uplus \{\star\}, \star) \to (Y \uplus \{\star\}, \star) \\
    &= x \mapsto \begin{cases}
      \star \quad& \text{if }f(x)~\text{undefined} \\ 
      f(x) \quad& \text{otherwise.}
    \end{cases}
  \end{align*}
  Intuitively, the $\star$ stands in for the subset of $X$ on which $f$ is undefined.
  The inverse $G$ is also straightforward: its action on objects is to 
  forget remove point, $G((X, \star)) = X \setminus \star$. Then, its action on morphisms
  map the point to undefined:
  \begin{align*}
    G(g) &: X \setminus \star \to Y \setminus \star \\ 
    &= x \mapsto 
    \begin{cases}
      \bot \quad&\text{if }g(x) = \star \\
      g(x) \quad&\text{otherwise}
    \end{cases}
  \end{align*}

  Now, show that there is a natural isomorphism $\alpha : F \circ G \Rightarrow \id_{\textsf{FinSet}^*}$.
  This is not too hard and a good exercise. 
  It requires showing that there is a $\mathsf{FinSet}^*$ isomorphism 
  $(X, a) \cong ((X \setminus a) \uplus \star, \star)$. 

  \item Categorical equivalences are very powerful when they arise but 
  the condition is often too strong. 
  A natural weakening of this notion will be an \emph{adjunction}, which will be pairs of 
  functors between categories are a form of ``weak equivalence''. 

  \item The natural notion of weakening here will be as follows. Let $\calC$ and $\calD$ 
  be categories and $F : \calC \to \calD$ and $G : \calD \to \calC$ be functors. 
  Then, instead of requiring that round-trips of $F$ and $G$ be naturally isomorphic identity functors, 
  an adjunction will require instead that there simply exist some pair of natural transformations:
  \begin{align}
    \eta : \id_\calC \Rightarrow G \circ F  \qquad
    \epsilon :  F \circ G \Rightarrow \id_\calD 
    \label{eq:adjoint-units}
  \end{align}

  The natural transformation $\eta$ is called the \textbf{unit} of the adjunction and 
  $\epsilon$ is called the \textbf{counit}, the functor $F$ is called 
  the \textbf{left adjoint}, and the functor $G$ the \textbf{right adjoint}.

  \item There is one more coherence requirement we place on these natural
  transformations in order for them to form an adjunction. But, before we do
  that, let's examine this notion of notion of ``weak equivalence'' in a 
  simple preorder category to understand what it means before we give the 
  full definition of adjoint functors for general categories.
\end{itemize}

\section{Adjunctions for preorder categories: Galois connections}
\begin{itemize}
  \item As usual it's very useful to study new category theory definitions in 
  pre-order categories to gain order-theoretic intuition about them before 
  generalizing to the general setting where there can be more than one morphism 
  between any two objects.
  
  \item Let $(X, \le_X)$, $(Y, \le_Y)$ be preorders. Recall that a \emph{monotone map}
  is a function $f : X \to Y$ that respects the preorder, i.e., 
  if $x_1 \le_X x_2$ then $f(x_1) \le_Y f(x_2)$. Recall that functors between 
  preorder categories are exactly monotone maps.

  \item Let's interpret the definition of adjoint in this setting. First, 
  let's give two example functors between two preorder categories:

  % https://tikzcd.yichuanshen.de/#N4Igdg9gJgpgziAXAbVABwnAlgFyxMJZABgBpiBdUkANwEMAbAVxiRAB12BbOnACwDGjYABEAviDGl0mXPkIoAjOSq1GLNgEFJ0kBmx4CRAEwrq9Zq0QgAQjpkH5RAMxm1ltgGF7e2YYUkpIqqFhrWnDz8QgzAnhJSDnJGSkEh6lYgij76SQGmwebpbMbZfk4orgXuYSDOpY7JyAAsbqEZIvW5RC1VbWxNkqowUADm8ESgAGYAThBcSMogOBBIxgkgM3Or1MtIzuub84gArDsriABsB7NHF2dIAOzXW4iuS+cAHM9HD-eIAJzfBZ-O4gABGMDAUCQxCBlz+pnBkOhr1hukO23eSFBEKhezRUxuSFOWMQiNxKIJGyJiF+pK+6Jp-z+H2oFPxcLeuwBcNZpOZSLxqLEFDEQA
\begin{tikzcd}
  \mathcal{D} & A \arrow[r] \arrow[red, rd] & B \arrow[r] \arrow[red, dr, bend left] & C \arrow[red, d] \\
  \mathcal{C} & 1 \arrow[r] \arrow[blue, u] & 2 \arrow[r] \arrow[blue, ul, bend left] & 3 \arrow[blue, ul]
  \end{tikzcd}

  Here we've drawn $F : \calC \to \calD$ in blue and $G : \calD \to \calC$ in red.

  \item Let's check if these functors satisfy the adjoint requirement as we've seen it 
  so far. First, let's try to find a unit by finding a natural transformation
  $\eta : \id_\calC \Rightarrow G \circ F$. 
  Then, $\eta$ is a $\calC$-indexed family of morphisms in $\calC$:
  \begin{align*}
    \eta_1 = 1 \to 2 \quad \eta_2 = \id_2 \quad \eta_3 = \id_3
  \end{align*}

  We can also build the counit, which is a $\calD$-indexed family of morphisms in $\calD$:

  \begin{align*}
    \epsilon_D = \id_A \quad \epsilon_B = \id_B \quad \epsilon_C = B \to C 
  \end{align*}

  Intuitively, these two natural transformations are characterizing what happens 
  after round trips of following $F$ and $G$. The unit $\eta$ says that, after 
  following a round trip of $F$ and then $G$, we must end up ``in a higher place'' than we started;
  dually, the counit $\epsilon$ is saying that after a round trip of $G$ then $F$ 
  we must end up ``in a lower place'' than we started.

  \item Now we can reverse engineer an order-theoretic understanding for the 
  unit and the counit of the adjunctions $F$ and $G$.
  This relationship has a name in order theory:

  \begin{definition}[Galois connection]
    Let $(X, \le_X)$ and $(Y, \le_Y)$  be preorders and let $\alpha : X \to Y$ 
    and $\gamma : Y \to X$ be monotone functions. Then, $\alpha$ and 
    $\gamma$ form a \emph{Galois connection} if, for all $x \in X$ and $y \in Y$:
    \begin{align*}
      x \le \gamma(\alpha(x))  \qquad \text{and} \qquad \alpha(\gamma(y)) \le y
    \end{align*}
  \end{definition}

  \item Pause for a moment to absorb how close a Galois connection is to an inverse. 
  If $\gamma$ and $\alpha$ above were inverses, then indeed this would be 
  an inverse.

  \item An equivalent definition to the above is that, for all $x \in X, y \in Y$, 
  it holds that:
  \begin{align}
    \alpha(x) \le y \text{ if and only if } x \le \gamma(y)
    \label{eq:galois-hom}
  \end{align}

  Exercise: Why is this equivalence true? 

  \item A programming languages example: Galois connections are important in defining 
  abstract interpreters. The map $\alpha$ is called the \emph{abstraction function} and 
  $\gamma$ is called the \emph{concretization function}, and $\alpha$ is left adjoint to $\gamma$.

\end{itemize}

\subsection{Adjoints preserve structure}
\begin{itemize}
  \item An important property of adjoints is that left adjoints preserve colimits 
  and right adjoints preserve limits. This property is of course true for 
  Galois connections, so let's see what it means in that setting first:
  \begin{theorem}
    Let $(X, \le)$ and $(Y, \le)$ be complete partial orders (i.e., they both have all 
    meets and joins) and let $\alpha : X \to Y$ and $\gamma : Y
    \to X$ be monotone functions where $\alpha$ is left adjoint to $\gamma$. Then,
    for any $x_1, x_2 \in X$ we have that:
    \begin{align}
    \alpha(x_1 \sqcup x_1) = \alpha(x_1) \sqcup \alpha(x_2)
    \end{align}
    and, for any $y_1, y_2 \in Y$ we have that:
    \begin{align}
      \gamma(y_1 \sqcap y_2) = \gamma(y_1) \sqcap \gamma(y_2)
    \end{align}
  \end{theorem}
  \begin{proof}
    Let's prove that the left adjoint preserves meets; the other 
    one is a good exercise.
    By monotonicity, $\alpha(x_1) \sqcup \alpha(x_2) \le \alpha(x_1 \sqcup x_2)$. 
    So, we need to show $\alpha(x_1 \sqcup x_2) \le \alpha(x_1) \sqcup \alpha(x_2)$.
    We will use \cref{eq:galois-hom} to show this. 
    Let $m = x_1 \sqcup x_2$ and $m^* = \alpha(x_1)\sqcup \alpha(x_2)$. Then,
    \begin{align*}
      \alpha(m) \le m^* \iff m \le \gamma(m^*)
    \end{align*}
    Proceeding on the right hand side, 
    we use the fact that $\gamma$ is monotone to build an intermediate inequality:
    \begin{align*}
    \gamma(\alpha(x_1)) \sqcup \gamma(\alpha(x_2)) \le \gamma(\alpha(x_1)\sqcup \alpha(x_2))
    \end{align*}

    So, if we can show $x_1 \sqcup x_2 \le \gamma(\alpha(x_1)) \sqcup \gamma(\alpha(x_2))$, 
    we are done. This follows directly from adjoint closure: $x \le \gamma(\alpha(x))$ for all $x$.
  \end{proof}

\end{itemize}

\section{Adjunctions generally}

% \begin{itemize}
% \item We've seen three examples of ``inverses to product'':
%   exponentiation in STLC, magic wand in separation logic,
%   and lollipop in linear type theory.
%   These are all instances of adjunctions.
% \item Adjoints as a whole family of universal properties (recast univ. prop. of
%   (co)products, pullbacks, (co)limits more generally in these terms)
% \item Adjoints as approximate inverses (Galois connections in AI, comma category
%   statement. Example: biorthogonal closure)
% \item Right adjoints preserve limits, and left adjoints preserve colimits.
%     This immediately shows \begin{itemize}
%       \item In STLC, $\times$ distributes over $+$
%       \item Separating conjunction distributes over disjunction
%       \item Magic wand distributes over conjunction
%     \end{itemize}
% \item Adjoints are unique up to unique isomorphism. Application:
%   calculate exponents of presheaves
% \item Fun reference: \href{https://golem.ph.utexas.edu/category/2025/02/universal_characterization_of.html#c063905}{Meas is not cartesian closed}
% \end{itemize}
