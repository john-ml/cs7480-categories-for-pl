\chapter{Adjunctions}

\begin{itemize}
  \item So far we've mostly focussed on characterizing properties 
  of an individual category. Now let's broaden our focus to discuss 
  in more detail
  relationships between categories.
  \item First, let's study the strongest possible relationship between
  two categories: does it mean for two categories to be equivalent? 
  Following intuition from set isomorphism, we will want to design 
  two functors between the categories that behave like inverses.
  \item For this definition we will need a way of talking equivalence 
  of functors:
  \begin{definition}[Natural isomorphism of functors]
    Let $\calC$ and $\calD$ be categories and $F, G : \calC \to \calD$ 
    be functors. Then, $F$ and $G$ are \emph{naturally isomorphic},
    written $F \cong G$, if there is a natural transformation $\alpha : F \Rightarrow G$
    whose components are isomorphisms.

  \end{definition}
  % todo: make this just be X x Y

  % \begin{example}
  %   Let $\calC$ be the preorder category on natural numbers, and let 
  %   $\calD$ be a category whose objects are pairs of natural numbers 
  %   and where there is a morphism $(x,y) \to (z, w)$
  %   if and only if $x \le z$ or $y \le w$.

  %   Then, let $F : \calC \to \calD$ be $F = X \mapsto (0, X)$  and 
  %   $G : \calC \to \calD$ be $G  = X \mapsto (X, 0)$. Then,
  %   these two functors are naturally isomorphic along the natural 
  %   transformation
  %   $\alpha : F \Rightarrow G$ where components $\alpha_X : F(X) \to G(X)$
  %   swap the order of tules, $\alpha_X = (0, X) \to (X, 0)$.
  % \end{example}

  \item Now, to give an equivalence of categories, we must show that there exists a pair of functors $F$ and 
  $G$ between them whose compositions are naturally isomorphic to 
  the identity functors on each category:
  \begin{definition}[Equivalence of categories]
    \sloppy
    Let $\calC$ and $\calD$ be categories. Then, $\calC$ is equivalent
    to $\calD$, written $\calC \cong \calD$, if there are two functors 
    $F : \calC \to \calD$ and $G: \calC \to \calD$ satisfying
    $F \circ G \cong \id_\calC$ and $G \circ F \cong \id_\calD$, where $\id_\calC$ 
    and $\id_\calD$ are the identity functors on $\calC$ and $\calD$ respectively.
  \end{definition}
  \item There are some surprising categorical equivalences that are worth 
  thinking about:
  \begin{proposition}
    The category of finite pointed sets $\mathsf{FinSet}^*$ is equivalent to the category 
    of finite sets and partial functions $\mathsf{FinSet}_\text{par}$. 
  \end{proposition}

  The intuition behind this equivalence is interesting. First let's identify 
  the functors that form the equivalence. Let $F :
  \mathsf{FinSet}_\text{par} \to \mathsf{FinSet}^*$. The action 
  on objects is straightforward: $F(X) = (X \uplus \{\star\}, \star)$
  i.e., $F$ sends a set $X$ to a pointed set with $X$ extended by 
  some element $\star$ with $\star$ as the point.
  The action on morphisms is interesting. Let $f : X \to Y$ be a partial 
  function. Then, we define:
  \begin{align*}
    F(f) &: (X \uplus \{\star\}, \star) \to (Y \uplus \{\star\}, \star) \\
    &= x \mapsto \begin{cases}
      \star \quad& \text{if }f(x)~\text{undefined} \\ 
      f(x) \quad& \text{otherwise.}
    \end{cases}
  \end{align*}
  Intuitively, the $\star$ stands in for the subset of $X$ on which $f$ is undefined.
  The inverse $G$ is also straightforward: its action on objects is to 
  forget remove point, $G((X, \star)) = X \setminus \star$. Then, its action on morphisms
  map the point to undefined:
  \begin{align*}
    G(g) &: X \setminus \star \to Y \setminus \star \\ 
    &= x \mapsto 
    \begin{cases}
      \bot \quad&\text{if }g(x) = \star \\
      g(x) \quad&\text{otherwise}
    \end{cases}
  \end{align*}

  Now, show that there is a natural isomorphism $\alpha : F \circ G \Rightarrow \id_{\textsf{FinSet}^*}$.
  This is not too hard and a good exercise. 
  It requires showing that there is a $\mathsf{FinSet}^*$ isomorphism 
  $(X, a) \cong ((X \setminus a) \uplus \star, \star)$. 

  \item Categorical equivalences are very powerful when they arise but 
  the condition is often too strong. 
  A natural weakening of this notion will be an \emph{adjunction}, which will be pairs of 
  functors between categories are a form of ``weak equivalence''. 

  \item The natural notion of weakening here will be as follows. Let $\calC$ and $\calD$ 
  be categories and $F : \calC \to \calD$ and $G : \calD \to \calC$ be functors. 
  Then, instead of requiring that round-trips of $F$ and $G$ be naturally isomorphic identity functors, 
  an adjunction will require instead that there simply exist some pair of natural transformations:
  \begin{align}
    \eta : \id_\calC \Rightarrow G \circ F  \qquad
    \epsilon :  F \circ G \Rightarrow \id_\calD 
    \label{eq:adjoint-units}
  \end{align}

  The natural transformation $\eta$ is called the \textbf{unit} of the adjunction and 
  $\epsilon$ is called the \textbf{counit}, the functor $F$ is called 
  the \textbf{left adjoint}, and the functor $G$ the \textbf{right adjoint}.

  \item There is one more coherence requirement we place on these natural
  transformations in order for them to form an adjunction. But, before we do
  that, let's examine this notion of notion of ``weak equivalence'' in a 
  simple preorder category to understand what it means before we give the 
  full definition of adjoint functors for general categories.
\end{itemize}

\section{Adjunctions for preorder categories: Galois connections}
\begin{itemize}
  \item As usual it's very useful to study new category theory definitions in 
  pre-order categories to gain order-theoretic intuition about them before 
  generalizing to the general setting where there can be more than one morphism 
  between any two objects.
  
  \item Let $(X, \le_X)$, $(Y, \le_Y)$ be preorders. Recall that a \emph{monotone map}
  is a function $f : X \to Y$ that respects the preorder, i.e., 
  if $x_1 \le_X x_2$ then $f(x_1) \le_Y f(x_2)$. Recall that functors between 
  preorder categories are exactly monotone maps.

  \item Let's interpret the definition of adjoint in this setting. First, 
  let's give two example functors between two preorder categories:

  % https://tikzcd.yichuanshen.de/#N4Igdg9gJgpgziAXAbVABwnAlgFyxMJZABgBpiBdUkANwEMAbAVxiRAB12BbOnACwDGjYABEAviDGl0mXPkIoAjOSq1GLNgEFJ0kBmx4CRAEwrq9Zq0QgAQjpkH5RAMxm1ltgGF7e2YYUkpIqqFhrWnDz8QgzAnhJSDnJGSkEh6lYgij76SQGmwebpbMbZfk4orgXuYSDOpY7JyAAsbqEZIvW5RC1VbWxNkqowUADm8ESgAGYAThBcSMogOBBIxgkgM3Or1MtIzuub84gArDsriABsB7NHF2dIAOzXW4iuS+cAHM9HD-eIAJzfBZ-O4gABGMDAUCQxCBlz+pnBkOhr1hukO23eSFBEKhezRUxuSFOWMQiNxKIJGyJiF+pK+6Jp-z+H2oFPxcLeuwBcNZpOZSLxqLEFDEQA
\begin{tikzcd}
  \mathcal{D} & A \arrow[r] \arrow[red, rd] & B \arrow[r] \arrow[red, dr, bend left] & C \arrow[red, d] \\
  \mathcal{C} & 1 \arrow[r] \arrow[blue, u] & 2 \arrow[r] \arrow[blue, ul, bend left] & 3 \arrow[blue, ul]
  \end{tikzcd}

  Here we've drawn $F : \calC \to \calD$ in blue and $G : \calD \to \calC$ in red.

  \item Let's check if these functors satisfy the adjoint requirement as we've seen it 
  so far. First, let's try to find a unit by finding a natural transformation
  $\eta : \id_\calC \Rightarrow G \circ F$. 
  Then, $\eta$ is a $\calC$-indexed family of morphisms in $\calC$:
  \begin{align*}
    \eta_1 = 1 \to 2 \quad \eta_2 = \id_2 \quad \eta_3 = \id_3
  \end{align*}

  We can also build the counit, which is a $\calD$-indexed family of morphisms in $\calD$:

  \begin{align*}
    \epsilon_A = \id_A \quad \epsilon_B = \id_B \quad \epsilon_C = B \to C 
  \end{align*}

  Intuitively, these two natural transformations are characterizing what happens 
  after round trips of following $F$ and $G$. The unit $\eta$ says that, after 
  following a round trip of $F$ and then $G$, we must end up ``in a higher place'' than we started;
  dually, the counit $\epsilon$ is saying that after a round trip of $G$ then $F$ 
  we must end up ``in a lower place'' than we started.

  \item Now we can reverse engineer an order-theoretic understanding for the 
  unit and the counit of the adjunctions $F$ and $G$.
  This relationship has a name in order theory:

  \begin{definition}[Galois connection]
    Let $(X, \le_X)$ and $(Y, \le_Y)$  be preorders and let $\alpha : X \to Y$ 
    and $\gamma : Y \to X$ be monotone functions. Then, $\alpha$ and 
    $\gamma$ form a \emph{Galois connection} if, for all $x \in X$ and $y \in Y$:
    \begin{align*}
      x \le \gamma(\alpha(x))  \qquad \text{and} \qquad \alpha(\gamma(y)) \le y
    \end{align*}
  \end{definition}

  \item Pause for a moment to absorb how close a Galois connection is to an inverse. 
  If $\gamma$ and $\alpha$ above were inverses, then indeed this would be 
  an inverse.

  \item An equivalent definition to the above is that, for all $x \in X, y \in Y$, 
  it holds that:
  \begin{align}
    \alpha(x) \le y \text{ if and only if } x \le \gamma(y)
    \label{eq:galois-hom}
  \end{align}

  Exercise: Why is this equivalence true? 

  \item A programming languages example: Galois connections are important in defining 
  abstract interpreters. The map $\alpha$ is called the \emph{abstraction function} and 
  $\gamma$ is called the \emph{concretization function}, and $\alpha$ is left adjoint to $\gamma$.



\end{itemize}

\subsection{Adjoints preserve structure}
\begin{itemize}
  \item An important property of adjoints is that left adjoints preserve colimits 
  and right adjoints preserve limits. This property is of course true for 
  Galois connections, so let's see what it means in that setting first:
  \begin{theorem}
    Let $(X, \le)$ and $(Y, \le)$ be complete partial orders (i.e., they both have all 
    meets and joins) and let $\alpha : X \to Y$ and $\gamma : Y
    \to X$ be monotone functions where $\alpha$ is left adjoint to $\gamma$. Then,
    for any $x_1, x_2 \in X$ we have that:
    \begin{align}
    \alpha(x_1 \sqcup x_1) = \alpha(x_1) \sqcup \alpha(x_2)
    \end{align}
    and, for any $y_1, y_2 \in Y$ we have that:
    \begin{align}
      \gamma(y_1 \sqcap y_2) = \gamma(y_1) \sqcap \gamma(y_2)
    \end{align}
  \end{theorem}
  \begin{proof}
    Let's prove that the left adjoint preserves meets; the other 
    one is a good exercise.
    By monotonicity, $\alpha(x_1) \sqcup \alpha(x_2) \le \alpha(x_1 \sqcup x_2)$. 
    So, we need to show $\alpha(x_1 \sqcup x_2) \le \alpha(x_1) \sqcup \alpha(x_2)$.
    We will use \cref{eq:galois-hom} to show this. 
    Let $m = x_1 \sqcup x_2$ and $m^* = \alpha(x_1)\sqcup \alpha(x_2)$. Then,
    \begin{align*}
      \alpha(m) \le m^* \iff m \le \gamma(m^*)
    \end{align*}
    Proceeding on the right hand side, 
    we use the fact that $\gamma$ is monotone to build an intermediate inequality:
    \begin{align*}
    \gamma(\alpha(x_1)) \sqcup \gamma(\alpha(x_2)) \le \gamma(\alpha(x_1)\sqcup \alpha(x_2))
    \end{align*}

    So, if we can show $x_1 \sqcup x_2 \le \gamma(\alpha(x_1)) \sqcup \gamma(\alpha(x_2))$, 
    we are done. This follows directly from adjoint closure: $x \le \gamma(\alpha(x))$ for all $x$.
  \end{proof}
\end{itemize}

\section{Adjunctions for preorders, continued}
\marginnote{Notes from John's lecture}
\begin{itemize}
  \item Recall the Galois connection example:

  \begin{center}
  \includegraphics[width=300px]{fig/galois.png}
  \end{center}
  \item We defined a pair of functions $\alpha : \calC \to \calA$
  and $\gamma : \calD \to \calC$, where $\alpha$ is 
  left-adjoint to $\gamma$.
  We defined an example $\gamma$ as follows (shown in red):
   \begin{center}
  \includegraphics[width=300px]{fig/galois-2.png}
  \end{center}
 
  \item To get to the general definition of preorder,
  we will study the properties of $\alpha$ 
  and $\gamma$. We will want our general definition of 
  adjoint to preserve \emph{all} of this structure

  \item What is the property we want $\alpha$ to have, 
  given this definition of $\gamma$? 
  In some sense, it is \emph{the most precise 
  abstraction} would can make.

  \item For example, we can define a really bad $\alpha$ 
  that is very imprecise. For instance, 
  we could let $\alpha(\_) = ?$. But if we did this, 
  then $\alpha$ would \emph{not be an adjoint}.

  \item There is a collection of interesting properties 
  we want $\alpha$ to have so that it is in some 
  sense ``most precise'':
  \begin{enumerate}
    \item $\alpha(c)$ is the smallest $a \in \calA$ 
    such that $c \le \gamma(a)$. 
    \item $\gamma(a)$ is the largest $c \in \calC$ 
    such that $\alpha(c) \le a$.
    \item For all $a \in \calA$ and $c \in \calC$, 
    $\alpha(c) \le a$ if and only if 
    $c \le \gamma(a)$.
  \end{enumerate}

  \item These fun facts are more than just fun facts: they are 
  in fact \emph{equivalent definitions} of adjoints (at least 
  for preorders). 

  \begin{theorem}
    Let % https://tikzcd.yichuanshen.de/#N4Igdg9gJgpgziAXAbVABwnAlgFyxMJZABgBpiBdUkANwEMAbAVxiRAB12BbOnACwDGjYAEEAviDGl0mXPkIoAjOSq1GLNpx78hDYAGEJY1TCgBzeEVAAzAE4QuSMiBwQkykACMYYKEgDMzvTMrIgc7IxofHSS0iB2Du7Urk7U3r4BQeqh4WZ0XDySFGJAA
    \begin{tikzcd}
    \mathcal{C} \arrow[r, "\alpha", bend left] & \mathcal{A} \arrow[l, "\gamma", bend left]
    \end{tikzcd} be a pair of functors. Then the following 
    are equivalent:
    \begin{enumerate}
      \item $\alpha \dashv \gamma$ 
      \item For any $c \in \calC$, $\alpha(c)$ is the 
      smallest $a \in \calA$ such t hat $c \le \gamma(a)$
      \item For any $a \in \calA$, 
$\gamma(a)$ is the largest $c \in \calC$ 
    such that $\alpha(c) \le a$.
    \item For all $a \in \calA$ and $c \in \calC$, 
    $\alpha(c) \le a$ if and only if 
    $c \le \gamma(a)$.
    \end{enumerate}
  \end{theorem}

  \item This is going to be a big proof. First we'll prove 
  that (1) implies (2).

  \todo{} fell behind. Same as the next one really.

  \item If $\alpha \dashv \gamma$ then for all 
  $a \in \calA$, $\gamma(a)$ is the largest 
  $c \in \calC$ such that $\alpha(c) \le a$

  First, $\alpha(\gamma(a)) \le a$, holds by 
  $\alpha \dashv a$

  Now we need to show $\gamma(a)$ is the largest such. 
  Suppose $c \in \calC$ such that $\alpha(c) \le a$. 
  Need to show that $c \le \gamma(a)$. 
  By $\gamma$ is monotone, 
  \begin{align*}
    \gamma(\alpha(c)) \le \gamma(a)
  \end{align*}
  By definition of adjoint, $c \le \gamma(\alpha(c))$, 
  which completes the proof.

  \item Proof that (2) implies (4). 

    For any $c \in \calC$, $\alpha(c)$ is the 
    smallest $a \in \calA$ such t hat $c \le \gamma(a)$.
    Want to show that for all $a \in \calA$, $c \in \calC$, 
    $\alpha(c) \le a \iff c \le \gamma(a)$.

    $(\Rightarrow)$: Have that $\alpha(c) \le a$, 
    need to show that $c \le \gamma(a)$.
    By monotonicity, we have that $\gamma(\alpha(c)) \le \gamma(a)$.
    By assumption, $c \le \gamma(\alpha(c))$.\footnote{Note: 
    cannot appeal to adjoint roundtrip property here. This is the assumption 
    in equivalent condition (2).}

    $(\Leftarrow)$: Have $ c \le \gamma(a)$. Want $\alpha(c) \le a$.
    Define some helpful notation: the \emph{downset} $c \downarrow \gamma = \{a
    \mid c \le \gamma(a) \}$. 
    Let $\varphi(a) = \text{min}(c \downarrow \gamma)$. 
    \todo {get the rest}

    \item Proof that (4) implies (1).
    If for all $a \in \calA, c \in \calC$ 
    it holds that $\alpha(c) \le a$ iff $c \le \gamma(a)$, 
    then
    for all $a \in \calA, c \in \calC$ 
    it holds that $\alpha(\gamma(a)) \le a$ and 
    $c \le \gamma(\alpha(c))$.


    First, show $\alpha(\gamma(a)) \le a$. 
    By assumption this is true if and only if 
    $\gamma(a) \le \gamma(a)$, which vacuously holds.
    The other side is identical.
\end{itemize}

\section{Generalizing beyond preorders}
\begin{itemize}
  \item We want a general definition of adjoints that generalizes the above 
  four-way equivalence. 
  \item To get there, we will first become more formal about 
  this downarrow.
  \item Let $\calC$ and $\calA$ be categories. 
  Then, for an object $c \in \calC$, 
  the category $c \downarrow G$ is a category 
  whose objects are morphisms $c \to G(a)$ for each 
  object $a \in \calA$.
  Morphisms are commuting triangles:\marginnote{This is an 
  instance of a \emph{comma category}.
  Thing of note: notice how the preorder notion of a 
  ``downset'' uses the same notation as comma category.}

  \begin{center}
    % https://tikzcd.yichuanshen.de/#N4Igdg9gJgpgziAXAbVABwnAlgFyxMJZABgBpiBdUkANwEMAbAVxiRAGMQBfU9TXfIRQBGclVqMWbAOIAKOgEpuvEBmx4CRUcPH1mrRCDl0A5Eq7iYUAObwioAGYAnCAFskZEDghJRE-WwOyo4u7oh+3kgATNR6UoZy1uYqzm4e1JGIMf7xIA4m3BRcQA
\begin{tikzcd}
  c \arrow[r, "f"] \arrow[rd, "f'"] & G(a) \arrow[d, "G(g)"] \\
                                    & G(a')                 
  \end{tikzcd}
  \end{center}
  where $c \mor{f} G(a)$ and $c \mor{f'} G(a')$ are two 
  objects.


  \item Now that we have a category we have a notion of ``smallest 
  element'' that generalizes categorically. 
  This means: the categorically generalization of property (2) 
  will be that for all objects $c \in \calC$, 
  the category $c \downarrow G$ has an initial object.

  This is a \emph{definition of adjoint}.
  


  \item Generalizing (3), we need to build a category $F \downarrow a$.
  This category has objects $F(c) \mor{f} a$ for 
  objects $c$ and morphisms $f$. 
  Then, a morphism in $F \downarrow a$ is again a triangle:

  \begin{center}
    % https://tikzcd.yichuanshen.de/#N4Igdg9gJgpgziAXAbVABwnAlgFyxMJZABgBpiBdUkANwEMAbAVxiRADEAKAYwEoQAvqXSZc+QigCM5KrUYs2dQcJAZseAkTKTZ9Zq0QceAcn4DZMKAHN4RUADMAThAC2SAEzUcEJMSEPnN0RPEG8kaTl9NntjZQDXXy8fRAi9BUN7QQoBIA
\begin{tikzcd}
  F(c) \arrow[r, "f"]              & a \\
  F(c') \arrow[u] \arrow[ru, "f'"] &  
  \end{tikzcd}
  \end{center}
  

  \item Let 
  % https://tikzcd.yichuanshen.de/#N4Igdg9gJgpgziAXAbVABwnAlgFyxMJZABgBpiBdUkANwEMAbAVxiRAB12BbOnACwDGjYAGEAviDGl0mXPkIoAjOSq1GLNpx78hDYAEEJY1TCgBzeEVAAzAE4QuSMiBwQkytc1aIQAMRDUAEYwYFBIAMzEUjb2jogerk5BIWGIkdT0XmwA4pIUYkA
\begin{tikzcd}
  \mathcal{C} \arrow[r, "F", bend left] & \mathcal{A} \arrow[l, "G", bend left]
  \end{tikzcd}

  Finally generalizing (1) is a bit tricky. 
  Let $\eta : \id \Rightarrow G \circ F$ 
  and $\varepsilon : F \circ G \Rightarrow \id$. 
  Then, we have \emph{coherence conditions}\footnote{
    The idea of a coherence condition is essentially 
    ``equations you add to your definitions when you generalize 
    from preorders to categories''. Here, we've introduced 
    extra ways to get between $\calC$ and $\calA$ via 
    the units. This could add lots of extra morphisms 
    to our category if we aren't careful. So, the coherence 
    conditions restrict this proliferation of morphisms. 
    It's useful to think about what happens without these 
    coherence conditions; the easiest is to see that 
    initiality is violated in the category $c \downarrow G$.
  }

  \item The final generalization is (4). Categorically,
  this says for all $c \in \calC$ and $a \in \calA$, 
  $\calA(F(c), a) \cong \calC(c, G(a))$ that is natural in 
  $a$ and $c$.

  What does ``natural in $a$ and $c$'' mean? Well, we need two functors with
  compatible domains.  Intuition: take all the things that the definitions says 
  are natural and ``poke a hole in them''. Here are the holes: $\calA(F(-_1),
  -_2) : \calC^\text{op} \times \calA \to \mathsf{Set}$ and 
  $\calC(-_1, G(-_2)) : \calC^\text{op} \times \calA \to \text{Set}$.
  Then, the definition says there must be a natural 
  isomorphism between these two functors.

  \item Packaging all these definitions up:

\end{itemize}

\begin{fullwidth}
   \begin{definition}[Adjunction]
    Let % https://tikzcd.yichuanshen.de/#N4Igdg9gJgpgziAXAbVABwnAlgFyxMJZABgBpiBdUkANwEMAbAVxiRAB12BbOnACwDGjYAGEAviDGl0mXPkIoAjOSq1GLNpx78hDYABEJY1TCgBzeEVAAzAE4QuSMiBwQkykACMYYKEgDMzvTMrIggAGKS0iB2Du7Urk7U3r4Bzgx03gwACrJ4BGwMMNY4INTBGmEA4pIUYkA
    \begin{tikzcd}
    \mathcal{C} \arrow[r, "F", bend left] & \mathcal{D} \arrow[l, "G", bend left]
    \end{tikzcd}
    be a pair of functors. Then, the functor $F$ is \emph{left adjoint} to $G$, written $F \dashv G$, 
    if any of the following equivalent definitions hold:
    \begin{itemize}
      \item \emph{Unit \& co-unit}: There is a pair of natural transformations $\eta : \id_\calC \Rightarrow G \circ F$ 
      and $\epsilon : F \circ G \Rightarrow \id_\calD$ that satisfy the \emph{triangle identities}:
      \begin{center}
       % https://tikzcd.yichuanshen.de/#N4Igdg9gJgpgziAXAbVABwnAlgFyxMJZABgBpiBdUkANwEMAbAVxiRADEQBfU9TXfIRQBGclVqMWbdgHFOPPtjwEio4ePrNWiDt3EwoAc3hFQAMwBOEALZIyIHBCSiQDLGG0goEJgCMGrNQAFjB0UEhgTAwM1Dh0WAxskB4g1JpSOuwABAA6OTBxWdy8IJY2zrFOiABM1G4pOt5+AakgIWERUTEO8Yk6yYESWmx5MGjYDARZ8iVltoj2jki1ru6eTf6D7eGIkdGxvUkEg+mewgD68hRcQA
\begin{tikzcd}
  F \arrow[r, "F \eta ", Rightarrow] \arrow[rd, "1_F", Rightarrow] & FGF \arrow[d, "\epsilon F", Rightarrow] \\
                                                                   & F                                      
  \end{tikzcd} 
  \qquad
% https://tikzcd.yichuanshen.de/#N4Igdg9gJgpgziAXAbVABwnAlgFyxMJZABgBpiBdUkANwEMAbAVxiRAHEQBfU9TXfIRQBGclVqMWbdgDFOPPtjwEio4ePrNWiDt3EwoAc3hFQAMwBOEALZIyIHBCSiQDLGG0goEJgCMGrNQAFjB0UEhgTAwM1Dh0WAxskB4g1JpSOuwABAA6OTBxWdy8IJY2zrFOiABM1G4pOt5+AakgIWERUTEO8Yk6yYESWmx5MGjYDARZ8iVltoj2jki1ru6eTf6D7eGIkdGxvUkEg+mewgD68hRcQA
\begin{tikzcd}
  G \arrow[r, "G \eta ", Rightarrow] \arrow[rd, "1_G", Rightarrow] & GFG \arrow[d, "\epsilon G", Rightarrow] \\
                                                                   & G                                      
  \end{tikzcd}
      \end{center}
      \item \emph{Natural isomorphism of hom-sets:} For any objects $c \in \calC$ and $d \in \calD$
      there is an isomorphism:
      \begin{align*}
        \calD(Fc, d) \cong \calC(c, Gd)
      \end{align*}
      that is \emph{natural in $c$ and $d$}, i.e., there is a natural isomorphism 
      between the two functors:
      
      \begin{center}
       % https://tikzcd.yichuanshen.de/#N4Igdg9gJgpgziAXAbVABwnAlgFyxMJZABgBpiBdUkANwEMAbAVxiRAB12BbOnACwDGjYAGEAvgD1OOGAA8cwCGjEACaVi7w13XoOEARMSDGl0mXPkIoAjOSq1GLNtLkKAyjBxGx9mFADm8ESgAGYAThBcSGQgOBBItiAARjBgUEgAzDH0zKyIHDr8QgzAhgAUAGIAtKRVAJTGpiDhkdHUcQnUKWlIVVnUDHQpDAAK5ngEbGFY-nw4INQ5TvmcPEXC4mU1KgDi9cYUYkA
\begin{tikzcd}
  \mathcal{C}^\text{op} \times \mathcal{D} \arrow[r, "{\mathcal{D}(F-,-)}", bend left] \arrow[r, "{\mathcal{C}(-, G-)}"', bend right] & \text{Set}
  \end{tikzcd}
      \end{center}
      
      \item \emph{Initiality condition}: For every object $c \in \calC$, the category $c \downarrow G$ 
      has an initial object.

      \item \emph{Terminality condition}: For every object $d \in \calD$, the category $F \downarrow d$ 
      has a terminal object.

    \end{itemize} 
  \end{definition}
 
\end{fullwidth}



\section{Examples}
\begin{itemize}
  \item A category $\calC$ has terminal objects 
  if and only if 
  the there is a functor $F : \calC \to 1$ (1 is the 1-object category)
  that has a right adjoint.
  \item 
\end{itemize}

% \begin{itemize}
% \item We've seen three examples of ``inverses to product'':
%   exponentiation in STLC, magic wand in separation logic,
%   and lollipop in linear type theory.
%   These are all instances of adjunctions.
% \item Adjoints as a whole family of universal properties (recast univ. prop. of
%   (co)products, pullbacks, (co)limits more generally in these terms)
% \item Adjoints as approximate inverses (Galois connections in AI, comma category
%   statement. Example: biorthogonal closure)
% \item Right adjoints preserve limits, and left adjoints preserve colimits.
%     This immediately shows \begin{itemize}
%       \item In STLC, $\times$ distributes over $+$
%       \item Separating conjunction distributes over disjunction
%       \item Magic wand distributes over conjunction
%     \end{itemize}
% \item Adjoints are unique up to unique isomorphism. Application:
%   calculate exponents of presheaves
% \item Fun reference: \href{https://golem.ph.utexas.edu/category/2025/02/universal_characterization_of.html#c063905}{Meas is not cartesian closed}
% \end{itemize}
