\chapter{Elementary topoi}
\marginnote{This chapter is heavily based on \citep{lawvere2009conceptual} chapters 32 and 33. 
This chapter is work-in-progress.}

\begin{itemize}
    \item A \emph{topos} is a category whose internal language is like set theory.
    If you like, you can think of it as a ``\texttt{\#lang} for math'': it gives a 
    reinterpretation internal to some category of all the mathematical symbols like $\land$, $\lor$, $\exists$,
    $\forall$, $\Rightarrow$, set-builder notation, etc.

    \item Many of the categories we've been discussing so far --
    presheaf categories, the category of $G$-sets, the category of 
    finite transition systems -- are examples of toposes.

    \item Today we'll discuss elementary toposes, which are toposes whose 
    internal language is intuitionistic higher order logic.  This means that the
    law of excluded middle and the axiom of choice may not hold in these topoi
     (i.e., logical statements like $\neg \neg \varphi$ may not entail $\varphi$) 

    \item To get there, we will study the critical set-theoretic relationship of 
    a \emph{subset}, which is a special relationship between two objects that 
    says one object a \emph{part} of another. Once we have the categorification of 
    subset in hand, we will use it to define other logical connectives.
\end{itemize}

\section{Parts of objects}
\begin{itemize}
    \item Recall the definition of a monomorphism,
    which is the categorification of an inclusion map (or injective function):
    from the homework:
    \begin{definition}[Monomorphism]
        Let $\calC$ be a category. A morphism $X \mor{f} Y$ is a \emph{monomorphism}
        if for any morphisms 
        % https://tikzcd.yichuanshen.de/#N4Igdg9gJgpgziAXAbVABwnAlgFyxMJZABgBpiBdUkANwEMAbAVxiRAC0QBfU9TXfIRQBGclVqMWbABrdxMKAHN4RUADMAThAC2SMiBwQkokHAAWWNTiQBaAEzV6zVohCKA+sO68QmnXupDY2pzS2tEEycpVw87EGoGOgAjGAYABX48AjYNLEUzay4KLiA
\begin{tikzcd}
    Z \arrow[r, "g_1", shift left=2] \arrow[r, "g_2"', shift right] & X
    \end{tikzcd}, 
    if $f \circ g_1 = f \circ g_2$ then $g_1 = g_2$. This property is called \emph{left cancellation} of $f$.
    Monomorphisms are typically drawn with a hook arrow like 
    $X \hookrightarrow Y$.
    \end{definition}

    \item In the homework we saw that, in the category of finite sets, a 
    function is injective if and only if it is a monomorphism. 
    Injective maps identify subsets, so monomorphisms will identify 
    subobjects. 

    \item Monomorphisms give us a natural notion of a categorification of subset, which will 
    be called a \emph{part of an object}:
    \begin{definition}[Part of an object]
        Let $\calC$ be a category and $X$ be an object in $\calC$. Then, $Y$ 
        is a \emph{part} of $X$ if there is a monomorphism $Y \hookrightarrow X$.
    \end{definition}

%     \begin{definition}[Subobject]
%         Let $\calC$ be a category and $X$ be an object in $\calC$. Then, 
%         a \emph{subobject of $X$} is an equivalence class of 
%         monomorphisms into $X$ where two 
%         monomorphisms $h : Y \hookrightarrow X$ and $k : Z \hookrightarrow X$ 
%         are equivalent if there is an isomorphism $v : Y \to Z$ 
%         such that the following triangle commutes:
%         % https://tikzcd.yichuanshen.de/#N4Igdg9gJgpgziAXAbVABwnAlgFyxMJZARgBoAGAXVJADcBDAGwFcYkQANEAX1PU1z5CKchWp0mrdgE0efEBmx4CRUcXEMWbRCABaPcTCgBzeEVAAzAE4QAtkjIgcEJKKf0sjdgAsIEANYgNJpSOt5yljb2iABMNM6u8R5eYX6BwZLaIIG8kXZIcU4uiI4hWTmU3EA
% \begin{tikzcd}
%     Y \arrow[r, "h", hook]                 & X \\
%     Z \arrow[ru, "k", hook] \arrow[u, "v"] &  
%     \end{tikzcd}
%     \end{definition}
 
    % \item A natural question one might ask is: why not have the simpler 
    % definition where a subobject is simply a monomorphism? 

    % \item You can think of $Z$ in the above definition as a ``test object'': in 
    % order to show that $f$ is a monomorphism, we have to show that \emph{for every 
    % other test object $Z$ in the category $\calC$}, it is the case that $f$ satisfies 
    % the left cancellation property for any pair of morphisms from $Z$ to $X$.

    % \item In many categories it isn't necessary to use 

\end{itemize}

\section{Subobject classifier}
\begin{itemize}
    \item The subobject classifier is the categorification of a 
    characteristic function.

    \begin{definition}[Characteristic function]
        Let $S$ be a set and $A \subseteq S$. 
        Then, the \emph{characteristic function of $A$}, written $\chi_A : A \to \{\top, \bot\}$, is the function:
        \begin{align*}
            \chi_A = x \mapsto \begin{cases}
                \top \quad&\text{if } x \in A \\ 
                \bot \quad&\text{otherwise.}
            \end{cases}
        \end{align*}
    \end{definition}

    
    \item Intuitively, characteristic functions ought to be in bijection with the 
    subsets they are defined over. We can see this as follows. Suppose we have the following 
    diagram in the category of finite sets:
    
    \begin{center}
        % https://tikzcd.yichuanshen.de/#N4Igdg9gJgpgziAXAbVABwnAlgFyxMJZABgBoBGAXVJADcBDAGwFcYkQBlEAX1PU1z5CKchWp0mrdgB1pwWTghpSAAlkAjCDlncefEBmx4CRUcXEMWbRCFnzpcHPQBOOnuJhQA5vCKgAZs4QALZIAEw0ikiiElbsOM5SvAFBoYhkIFGIMZZSNrIAxgAWWAD6AILu3EA
\begin{tikzcd}
    & \{\star\} \arrow[d, "true"] \\
S \arrow[r, "\chi_A"] & {\{\top, \bot\}}           
\end{tikzcd}
    \end{center}
    
    Define $true : \star \mapsto \top$. Then, there is a set $B$
    that is the pullback along the morphism $! : B \to \star$ 
    and $\chi_A$, and this set $B$ must be isomorphic to $A$.

    \item This setup lends itself naturally to a categorical definition:

    \begin{definition}[Subobject classifier]
        \sloppy
        Let $\calC$ be a category with terminal objects and $\Omega$ 
        be an object in $\calC$.
        Then, a morphism $true : \mathbf{1} \to \Omega$
        is called a \emph{subobject classifier of $\calC$} if for 
        every monomorphism $j : U \hookrightarrow X$ there is a unique 
        morphism $\chi_j : X \to \Omega$ called the \emph{classifying morphism for 
        the subobject represented by $j$} such that the following diagram:
        \begin{center}
 % https://tikzcd.yichuanshen.de/#N4Igdg9gJgpgziAXAbVABwnAlgFyxMJZABgBoBGAXVJADcBDAGwFcYkQANEAX1PU1z5CKchWp0mrdgB1pAeQC2MAOb0efEBmx4CRUcXEMWbRCHLr+2oUTIGaRqaYCqPcTCjL4RUADMAThAKSADMNDgQSGQgOPRYjOwAFhAQANYg9pImIABW6SCM9ABGMIwACgI6wvkwPjgWIP6BSABMYRGIohLG7Dh+Ury+AUGIUeFInQ5ZsgDGCVgA+rkDDUMhbS0Z3aYAhK7cQA
\begin{tikzcd}
    U \arrow[d, "j", hook] \arrow[r, "!"] & 1 \arrow[d, "true"] \\
    X \arrow[r, "\chi_j"]                 & \Omega             
    \end{tikzcd}
        \end{center}
        is a pullback along $\chi_j$ and $!$. In this situation, the object $\Omega$ is called 
        the \emph{object of truth values}.
    \end{definition}

    \item In the category of finite sets, the two-element set $\{\top, \bot\}$ 
    is the object of truth values and the map $true : \{\star\} \to \top$ is 
    the subobject classifier.

    \item The subobject classifier can be much more interesting in different 
    categories!
    Let's see an example of one.

    \begin{definition}[Topos of trees]
        Consider the following category called $\mathbb{N}_\rightarrow$:
        \begin{center}
            % https://tikzcd.yichuanshen.de/#N4Igdg9gJgpgziAXAbVABwnAlgFyxMJZABgBpiBdUkANwEMAbAVxiRGJAF9T1Nd9CKAIzkqtRizZCuPEBmx4CRAEyjq9Zq0QhlM3goFEAzGvGa2RvXL6LByACymNk7QDp3XMTCgBzeEVAAMwAnCABbJBMQHAgkZW4g0IjEVWjYxCEEkBDwpBE0pGIsnOTHAsQjTgpOIA
\begin{tikzcd}
    0 \arrow[r] & 1 \arrow[r] & 2 \arrow[r] & 3 \arrow[r] & ...
    \end{tikzcd}
        \end{center}
    The \emph{topos of trees} is the category of presheaves on $\mathbb{N}_\leftarrow$
    written $\mathsf{Psh}(\mathbb{N}_\rightarrow)$.
    \end{definition}
    
    \item We haven't defined what a topos is yet, so just treat the above as the 
    definition for a special kind of presheaf category. But, let's see what a subobject 
    classifier is in this category.
    First, let's look at what monomorphisms look like. Here is an example of a 
    monomorphism:

    \begin{center}
        \includegraphics[width=200px]{fig/topos-1.png}
    \end{center}

    The red arrow is a natural transformation $Q \Rightarrow P$ where each 
    component of the natural transformation is an injective function.
    In a situation analogous to the one in finite sets, a natural transformation 
    $\alpha : P \Rightarrow Q$ is a monomorphism in a presheaf category if and only 
    if each component is an injective map.

    \item Objects of this category can be thought of as ``sets of facts that are true 
    at particular times''. They will correspond to \emph{Kripke models}, and 
    a monomorphism will correspond to a Kripke sub-world.

    % \item Syntax of modal logic:
    % \begin{align*}
    %     \phi, \psi ::= \phi \land \psi \mid \phi \lor \psi \mid \neg \phi \mid \square \phi \mid \diamond \phi \mid x
    % \end{align*}

    % Intuition: $\diamond \varphi$ says ``there is an accessible world where $\varphi$ holds'',
    % and $\square \varphi$ says ``$\varphi$ always holds''.

    \item Now we can think about the object of truth values. It is this:

    \begin{center}
        \includegraphics[width=250px]{fig/truth-1.png}
    \end{center}

    Formally, $\Omega$ is a presheaf that maps each object to the 
    set  $\{\top, 1, 2, \cdots\}$, and each 
    morphism to the predecessor function that maps $1$ to $\top$ and $\top$ to $\top$.
    Intuitively, think of each element of each set as representing ``the number 
    of steps until something is true''. With this intuition, a non-zero value represents 
    that something is false at a particular time step.

    \item Given this object of truth values, we can design the subobject 
    classifier as taking each time step and mapping it to the number of 
    steps of ``gas'' that timestep has left:

    \begin{center}
        \includegraphics[width=250px]{fig/truth-2.png}
    \end{center}

    
    \item Let's use this object of truth values to find the classifying morphism for 
    $\alpha$ above. We need to find a morphism $P \mor{\chi_\alpha} \Omega$ 


    \item  The mere existence of subobject classifiers already forces your category 
    to behave in set-like ways. An example:
        Every category with a subobject classifier is \emph{balanced}, 
        meaning that if there is an epimorphism and monorphism between 
        two objects, they are isomorphic. This is just like in sets where 
        the existence of an injection and surjection implies an isomorphism.
    
\end{itemize}


\section{Generalized elements}

Important picture: the subobject classifier gives us an interpretation for set-builder.\marginnote{Think of $\varphi(x)$ 
as an open term with free variable of type $X$.}

\begin{center}
 % https://tikzcd.yichuanshen.de/#N4Igdg9gJgpgziAXAbVABwnAlgFyxMJZABgBpiBdUkANwEMAbAVxiRAB13gAPAAk6xheADX7sAtlihj6AJzQALLAApuASk4BfEJtLpMufIRRkAjFVqMWbYTr0gM2PASKnyF+s1aIQpu-qcjV1Jzak9rH04AeXEYAHM6HQsYKDj4IlAAM1kIcSQyEBwIJDdCuiwGNgUICABrfxBs3PzqIqQAJl0snLzEUrbEAGYwq28OdjlFFXUGpt721uKhka82HFlrTQpNIA
\begin{tikzcd}
    \{x \in X \mid \varphi(x)\} \arrow[d, hook] \arrow[r] & 1 \arrow[d, "true"] \\
    X \arrow[r, "\varphi(x)"]                             & \Omega             
    \end{tikzcd}
\end{center}

We can say \emph{an element of X satisfies $\varphi$} if we can find a 
morphism $x_0$ such that the following commutes:

\begin{center}
    % https://tikzcd.yichuanshen.de/#N4Igdg9gJgpgziAXAbVABwnAlgFyxMJZARgBoAGAXVJADcBDAGwFcYkQAdD4ADwAIuWMHwAaAjgFssUcQwBOaABZYAFDwCUXAL4gtpdJlz5CKMsWp0mrdiN36QGbHgJEATBQsMWbRCGJ2DJ2M3UnMaL2tfLgB5CRgAc3oAh0NnE2RyUM8rHz9dCxgoePgiUAAzOQgJJEyQHAgkMjr6LEZ2RQgIAGtkiqqamnqkVz1yyurEJqHEAGZwnPYueSVVDV7x4cGG2fnvdhw5a1GQPomAFi2a49OkC7rtpojcngB9cnytIA
\begin{tikzcd}
    & \{x \in X \mid \varphi(x)\} \arrow[d, hook] \arrow[r] & 1 \arrow[d, "true"] \\
1 \arrow[ru] \arrow[r, "x_0"] & X \arrow[r, "\varphi(x)"]                             & \Omega             
\end{tikzcd}
\end{center}

This is often too strong, e.g. Kripke semantics. Enter generalized elements:


We say $U \vDash \varphi$ (read ``$U$ forces $\varphi$'').


% \section{The category of parts of objects}
% \begin{definition}[Slice category]
% Let $\calC$ be a category 
% \end{definition}

\section{Kripke-Joyal Semantics}
% https://ncatlab.org/nlab/show/Kripke-Joyal+semantics
\begin{itemize}
    \item The Kripke-Joyal semantics gives us a full picture of working internal 
    to a topos
    \item These will look very familiar if you are familiar with separation logic 
    papers
    % \item Let $\calE$ be an elementary topos and $\alpha : U \to X$ be a generalized element 
% of $x \in \calE$.

\end{itemize}

\section{Elementary topos}
\begin{itemize}
    \item The Kripke-Joyal semantics tells us what features we need our 
    category to support, which will be an elementary topos:
    
\end{itemize}

\begin{definition}[Elementary topos]
    A category $\calC$ is an elementary topos if and only if
    it has products, coproducts, exponentials, an initial object, a terminal object, 
    an object of truth values, and satisfies that for every object $X$ 
    the slice category $\calC / X$ has products.
\end{definition}
\begin{itemize}
    \item Every presheaf category is an elementary topos (!!)
\end{itemize}

%%%%%%%%%%%%%%%%%%%%%%%%%%%%%%%%%%%%%%%%%%%%%%%%%%%%%%%%%%%%%%%%%%%%%%%%%%%%%%%%
\chapter{Sheaf semantics}
\marginnote{This chapter is heavily based on these notes from 
Alex Simpson: \url{https://alexksimpson.github.io/Talks/TutorialOnSheafSemantics.pdf}}

We will eventually get to sheaves. We will build up to it 
by first talking about Kripke semantics, then presheaf 
semantics, then we will talk about sheaves.

\section{Kripke semantics}

\begin{itemize}
    \item Kripke semantics are concerned with describing 
    beliefs over time.

\item Example: in the \emph{before time}, 
we believed that in the night sky there 
were 3 objects: a morning star, 
evenin star, and saturn.
Then, recently, we learned that in fact 
the morning star and evening star were 
both venus, and discovered a new planet 
Pluto. Then, in the present day, we learned
Pluto is not a planet:

\begin{center}
\includegraphics[width=200px]{fig/planet.png}
\end{center}

\item We can describe this situation using 
\emph{Kripke logic}, which has the following 
set of syntactic formulae:
\begin{align}
    \varphi, \psi ::= \top \mid \bot \mid \varphi \land \psi 
    \mid \varphi \lor \psi \mid \varphi \Rightarrow \psi \mid a
\end{align}
\item There is an interpretation function:
\begin{align}
    \lceil - \rceil : \text{Formula} \rightarrow 2^\mathbb{W}
\end{align}
where $\mathbb{W}$ is the set of \emph{Kripke worlds}, 
i.e. $\mathbb{W} = \{\text{before}, \text{recently}, \text{after}\}$.
The set of worlds is an ordered set ordered 
by a \emph{reachability relation $\le$}.
\item A subset $U$ of $\mathbb{W}$ is \emph{closed under} $\le$ 
if $w \in U$ and $w' \ge w$ then $w' \in U$.
\item Example: The predicate $\lceil$pluto is a planet$\rceil$ is not closed
under $\le$, since it is a planet in the Recently
time step but it is not a planet in the Now timestep.
This notion is called \emph{Kripke monotonicity}.
\item Now we can give the \emph{Kripke semantics} 
that tells us how to interpret formulae in terms of 
Kripke worlds. We will require that our semantics 
satisfies Kripke monotonicity.
\begin{align*}
    \lceil \top \rceil &= \mathbb{W} \\
    \lceil \bot \rceil &= \emptyset \\
    \lceil \varphi \land \psi \rceil &= \{w \mid w \in \lceil \varphi \text{ and } w \in \lceil \psi \rceil \} \\
    \lceil \varphi \lor \psi \rceil &= \{w \mid w \in \lceil \varphi \text{ or } w \in \lceil \psi \rceil \} \\
    \lceil \varphi \Rightarrow \psi \rceil 
    &= \{w \mid \forall w' \ge w \text{ if } w' \in \lceil \varphi \rceil \text{ then } w' \in \lceil \psi \rceil\} 
\end{align*}

\item These semantics are designed so that you have sound interpretations
of your proof rules: if $\varphi_1, \cdots, \varphi_n \vdash \psi$
then $\lceil \varphi_1 \land \cdots \land \varphi_n \rceil \subseteq \lceil \psi \rceil$,
where $\vdash$ is intuitionistic syntactic entailment.
\end{itemize}

\section{Presheaf semantics}
\begin{itemize}
    \item Now we can begin to generalize from the previous situation 
    to be categorical.
    \item The partial order of worlds $(\mathbb{W}, \le)$ becomes 
    a category $\calC$. 
    \item Continuing with our example, we have $\calC$ is:

    \begin{center}
       % https://tikzcd.yichuanshen.de/#N4Igdg9gJgpgziAXAbVABwnAlgFyxMJZABgBpiBdUkANwEMAbAVxiRACEYAzCAJ1YC+pdJlz5CKAIzkqtRizYAlGAGMYYHAwCeIISOx4CRAEwzq9Zq0QgAchADuu2TCgBzeEVBdeEALZIyEBwIJEk9EG8-UOpgpGMBCgEgA
\begin{tikzcd}
    Before \arrow[r] & Recently \arrow[r] & Now
    \end{tikzcd} 
    \end{center}

    \item A  monotone subset (i.e., a Kripke monotone predicate)
    is a presheaf $\calC^\text{op}$. For example, 
    we can have the following presheaf:

    \begin{center}
        \includegraphics[width=300px]{fig/planet-psh.png}
    \end{center}

\end{itemize}

\subsection{Formulas as subpresheaves}

\begin{itemize}
    \item Now we want to be able to interpret formulae $\varphi$ 
    with some free variables where $\Gamma \vdash \varphi$.
    As usual, we can associate the context $\Gamma$ with some 
    object $\lceil \Gamma \rceil$
    \item An entailment $\Gamma \vdash \varphi$ is a \emph{subpresheaf}, 
    which is a part of a presheaf that respects Kripke monotonicity.
    I.e., it is a monic natural transformation $U \to \lceil \Gamma \rceil$ 
    for some presheaf $U$

    \item For example, the following is a subpresheaf: \todo{}

    \item Now we can define a semantic interpretation $\lceil
    \rceil(w)$ as a collection of morphisms $w_1 \to w$
    that satisfies a category-theoretic analog of Kripke monotinicity 
    called \emph{generalized monotinicity:}
    if $(w' \mor{f} w) \in \lceil \Gamma \vdash \varphi \rceil(w)$
    and $w'' \mor{g} w'$, then the composite 
    $(w'' \mor{g} w \mor{f} w) \in \lceil \Gamma \vdash \varphi \rceil(w)$.

    \begin{align*}
        \lceil \top \rceil(w) &= \{ w' \mor{f} w \mid w' \text{ any object}\} \\
        \lceil \bot \rceil(w) &= \emptyset \\
        \lceil \varphi \land \psi \rceil &=
        \{f  : w' \to w \mid f \in \lceil \varphi \rceil (w) 
        \text{ and } 
        f \in \lceil \psi \rceil (w)\}\\
        \lceil \varphi \lor \psi \rceil &=
        \{f  : w' \to w \mid f \in \lceil \varphi \rceil (w) 
        \text{ or } 
        f \in \lceil \psi \rceil (w)\}
        \lceil \varphi \Rightarrow \psi \rceil &= \\
        \{f : w' \to w \mid \forall g : w'' \to w'. 
        \text{if } (w''' \mor{g} w' \mor{f} w) \in \lceil \varphi \rceil&
        \text{ then } (w''' \mor{g} w' \mor{f} w) \in \lceil \psi \rceil\}
    \end{align*}
    \item These are the \emph{generalized Kripke semantics}:
    these apply to any presheaf.
%     \item Implication is interesting:  you interpret 
%         $\lceil \varphi \Rightarrow \psi \rceil(w)$ as a monic 
%         map of type
%         % https://tikzcd.yichuanshen.de/#N4Igdg9gJgpgziAXAbVABwnAlgFyxMJZABgBpiBdUkANwEMAbAVxiRAFUQBfU9TXfIRQBGclVqMWbADrSGAYxhYGAAlkBxOgFstdNdIBOi5d3EwoAc3hFQAMwMQtSMiBwQko13ROIvJrhRcQA
% \begin{tikzcd}
%     U \arrow[r, tail] & \lceil \Gamma \rceil
%     \end{tikzcd}
%     We can reverse engineer what this monic map is by examining 
%     the order-theoretic definition:
%     \begin{align*}
%         \Big\{ w \mor{g} w_{cur} \mid \forall w' \mor{f} w . \text{if }
%         (w' \mor{f} w \mor{g} w_cur) \in \lceil \phi \rceil 
%         \text{ then } (w' \mor{f} w \mor{g} w_cur) \in \lceil \psi \rceil \Big\}
%     \end{align*}

\end{itemize}

\subsection{Example of presheaf semantics}
\begin{itemize}
    \item As an initial example, consider the topos of trees, i.e. presheaves on
    $(\mathbb{N}, \le)$. 
    \item Then, the interpretation of predicates $\lceil \varphi \rceil$ 
    are Kripke-monotone subsets of the natural numbers.

    \item A more interesting example is the presheaf semantics for 
    name generation. 

    \item We will take as our indexing category the category of 
    finite injective maps $\mathsf{FinInj}^\text{op}$. 
    Objects are finite sets $\Gamma$ and 
    homs are injection functions $\Gamma \to \Gamma'$.
    Intuitively, this represents how names can change over time:
    an injective map $\Gamma \to \Gamma'$ is some renaming.

    \item In the semantics, $\lceil \varphi \rceil (\Gamma)$ 
    is the set of renamings $r : \Gamma \to \Gamma'$ such 
    that $\varphi$ holds when $\Gamma'$ is ``available'', i.e., 
    $\Gamma$ is rewritten into some new context $\Gamma'$ that 
    validates $\psi$. 

    \item Notice that this is \emph{proof-relevant} in the sense 
    that the kind of substitution performed matters: there can be 
    many injective maps $\Gamma \to \Gamma'$
    
    \item As usual the interesting case here is implication,
    which can be read off directly from the Kripke semantics above.
    
    \item Just like in the Kripke semantics, the presheaf 
    semantics for Kripke logic will also validate intuitionistic 
    logic entailments.
\end{itemize}


\section{Sheaves}

\begin{itemize}
    \item At this point we've hit most of the common PL applications.
    Presheaf semantics get you very far: it is not that common to 
    need to use sheaves.
    \item The definition of sheaf semantics will depend on a notion 
    of \emph{coverage}. We can state the semantics first and then 
    explain what a coverage is after. These definitions are nearly 
    the same as presheaf semantics, but we only update the 
    portions that affect positive types (negation, disjunction):

    \todo{fill this in}
    \begin{align*}
        \lceil \top \rceil(w) &= \{ w' \mor{f} w \mid w' \text{ any object}\} \\
        \lceil \bot \rceil(w) &= \{ f : w' \to w \mid \phi \text{ covers } w\} \\
        \lceil \varphi \land \psi \rceil &=
        \{f  : w' \to w \mid f \in \lceil \varphi \rceil (w) 
        \text{ and } 
        f \in \lceil \psi \rceil (w)\}\\
        \lceil \varphi \lor \psi \rceil &=
        \{f  : w' \to w \mid f \in \lceil \varphi \rceil (w) 
        \text{ or } 
        f \in \lceil \psi \rceil (w)\}
        \lceil \varphi \Rightarrow \psi \rceil &= 
        \{f : w' \to w \mid \forall g : w'' \to w'. 
        \text{if } (w'' \mor{g} w' \mor{f} w) \in \lceil \varphi \rceil&
        \text{ then } (w'' \mor{g} w' \mor{f} w) \in \lceil \psi \rceil\}
    \end{align*}

    \item Include picture from phone, nice analogy of topological
    picture of coverage

    \item Definition of coverage

    \item Example: a set of formulae $\{ \varphi_1, \cdots, \varphi_n \} \triangle \varphi$ 
    if and only if $\bigvee_i \varphi_i = \varphi$.
\end{itemize}