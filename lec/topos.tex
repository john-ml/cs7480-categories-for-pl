\chapter{Elementary topoi}
\marginnote{This chapter is heavily based on \citep{lawvere2009conceptual} chapters 32 and 33. 
This chapter is work-in-progress.}

\begin{itemize}
    \item A \emph{topos} is a category whose internal language is like set theory.
    If you like, you can think of it as a ``\texttt{\#lang} for math'': it gives a 
    reinterpretation internal to some category of all the mathematical symbols like $\land$, $\lor$, $\exists$,
    $\forall$, $\Rightarrow$, set-builder notation, etc.

    \item Many of the categories we've been discussing so far --
    presheaf categories, the category of $G$-sets, the category of 
    finite transition systems -- are examples of toposes.

    \item Today we'll discuss elementary toposes, which are toposes whose 
    internal language is intuitionistic higher order logic.  This means that the
    law of excluded middle and the axiom of choice may not hold in these topoi
     (i.e., logical statements like $\neg \neg \varphi$ may not entail $\varphi$) 

    \item To get there, we will study the critical set-theoretic relationship of 
    a \emph{subset}, which is a special relationship between two objects that 
    says one object a \emph{part} of another. Once we have the categorification of 
    subset in hand, we will use it to define other logical connectives.
\end{itemize}

\section{Parts of objects}
\begin{itemize}
    \item Recall the definition of a monomorphism,
    which is the categorification of an inclusion map (or injective function):
    from the homework:
    \begin{definition}[Monomorphism]
        Let $\calC$ be a category. A morphism $X \mor{f} Y$ is a \emph{monomorphism}
        if for any morphisms 
        % https://tikzcd.yichuanshen.de/#N4Igdg9gJgpgziAXAbVABwnAlgFyxMJZABgBpiBdUkANwEMAbAVxiRAC0QBfU9TXfIRQBGclVqMWbABrdxMKAHN4RUADMAThAC2SMiBwQkokHAAWWNTiQBaAEzV6zVohCKA+sO68QmnXupDY2pzS2tEEycpVw87EGoGOgAjGAYABX48AjYNLEUzay4KLiA
\begin{tikzcd}
    Z \arrow[r, "g_1", shift left=2] \arrow[r, "g_2"', shift right] & X
    \end{tikzcd}, 
    if $f \circ g_1 = f \circ g_2$ then $g_1 = g_2$. This property is called \emph{left cancellation} of $f$.
    Monomorphisms are typically drawn with a hook arrow like 
    $X \hookrightarrow Y$.
    \end{definition}

    \item In the homework we saw that, in the category of finite sets, a 
    function is injective if and only if it is a monomorphism. 
    Injective maps identify subsets, so monomorphisms will identify 
    subobjects. 

    \item Monomorphisms give us a natural notion of a categorification of subset, which will 
    be called a \emph{part of an object}:
    \begin{definition}[Part of an object]
        Let $\calC$ be a category and $X$ be an object in $\calC$. Then, $Y$ 
        is a \emph{part} of $X$ if there is a monomorphism $Y \hookrightarrow X$.
    \end{definition}

%     \begin{definition}[Subobject]
%         Let $\calC$ be a category and $X$ be an object in $\calC$. Then, 
%         a \emph{subobject of $X$} is an equivalence class of 
%         monomorphisms into $X$ where two 
%         monomorphisms $h : Y \hookrightarrow X$ and $k : Z \hookrightarrow X$ 
%         are equivalent if there is an isomorphism $v : Y \to Z$ 
%         such that the following triangle commutes:
%         % https://tikzcd.yichuanshen.de/#N4Igdg9gJgpgziAXAbVABwnAlgFyxMJZARgBoAGAXVJADcBDAGwFcYkQANEAX1PU1z5CKchWp0mrdgE0efEBmx4CRUcXEMWbRCABaPcTCgBzeEVAAzAE4QAtkjIgcEJKKf0sjdgAsIEANYgNJpSOt5yljb2iABMNM6u8R5eYX6BwZLaIIG8kXZIcU4uiI4hWTmU3EA
% \begin{tikzcd}
%     Y \arrow[r, "h", hook]                 & X \\
%     Z \arrow[ru, "k", hook] \arrow[u, "v"] &  
%     \end{tikzcd}
%     \end{definition}
 
    % \item A natural question one might ask is: why not have the simpler 
    % definition where a subobject is simply a monomorphism? 

    % \item You can think of $Z$ in the above definition as a ``test object'': in 
    % order to show that $f$ is a monomorphism, we have to show that \emph{for every 
    % other test object $Z$ in the category $\calC$}, it is the case that $f$ satisfies 
    % the left cancellation property for any pair of morphisms from $Z$ to $X$.

    % \item In many categories it isn't necessary to use 

\end{itemize}

\section{Subobject classifier}
\begin{itemize}
    \item The subobject classifier is the categorification of a 
    characteristic function.

    \begin{definition}[Characteristic function]
        Let $S$ be a set and $A \subseteq S$. 
        Then, the \emph{characteristic function of $A$}, written $\chi_A : A \to \{\top, \bot\}$, is the function:
        \begin{align*}
            \chi_A = x \mapsto \begin{cases}
                \top \quad&\text{if } x \in A \\ 
                \bot \quad&\text{otherwise.}
            \end{cases}
        \end{align*}
    \end{definition}

    
    \item Intuitively, characteristic functions ought to be in bijection with the 
    subsets they are defined over. We can see this as follows. Suppose we have the following 
    diagram in the category of finite sets:
    
    \begin{center}
        % https://tikzcd.yichuanshen.de/#N4Igdg9gJgpgziAXAbVABwnAlgFyxMJZABgBoBGAXVJADcBDAGwFcYkQBlEAX1PU1z5CKchWp0mrdgB1pwWTghpSAAlkAjCDlncefEBmx4CRUcXEMWbRCFnzpcHPQBOOnuJhQA5vCKgAZs4QALZIAEw0ikiiElbsOM5SvAFBoYhkIFGIMZZSNrIAxgAWWAD6AILu3EA
\begin{tikzcd}
    & \{\star\} \arrow[d, "true"] \\
S \arrow[r, "\chi_A"] & {\{\top, \bot\}}           
\end{tikzcd}
    \end{center}
    
    Define $true : \star \mapsto \top$. Then, there is a set $B$
    that is the pullback along the morphism $! : B \to \star$ 
    and $\chi_A$, and this set $B$ must be isomorphic to $A$.

    \item This setup lends itself naturally to a categorical definition:

    \begin{definition}[Subobject classifier]
        \sloppy
        Let $\calC$ be a category with terminal objects and $\Omega$ 
        be an object in $\calC$.
        Then, a morphism $true : \mathbf{1} \to \Omega$
        is called a \emph{subobject classifier of $\calC$} if for 
        every monomorphism $j : U \hookrightarrow X$ there is a unique 
        morphism $\chi_j : X \to \Omega$ called the \emph{classifying morphism for 
        the subobject represented by $j$} such that the following diagram:
        \begin{center}
 % https://tikzcd.yichuanshen.de/#N4Igdg9gJgpgziAXAbVABwnAlgFyxMJZABgBoBGAXVJADcBDAGwFcYkQANEAX1PU1z5CKchWp0mrdgB1pAeQC2MAOb0efEBmx4CRUcXEMWbRCHLr+2oUTIGaRqaYCqPcTCjL4RUADMAThAKSADMNDgQSGQgOPRYjOwAFhAQANYg9pImIABW6SCM9ABGMIwACgI6wvkwPjgWIP6BSABMYRGIohLG7Dh+Ury+AUGIUeFInQ5ZsgDGCVgA+rkDDUMhbS0Z3aYAhK7cQA
\begin{tikzcd}
    U \arrow[d, "j", hook] \arrow[r, "!"] & 1 \arrow[d, "true"] \\
    X \arrow[r, "\chi_j"]                 & \Omega             
    \end{tikzcd}
        \end{center}
        is a pullback along $\chi_j$ and $!$. In this situation, the object $\Omega$ is called 
        the \emph{object of truth values}.
    \end{definition}

    \item In the category of finite sets, the two-element set $\{\top, \bot\}$ 
    is the object of truth values and the map $true : \{\star\} \to \top$ is 
    the subobject classifier.

    \item The subobject classifier can be much more interesting in different 
    categories!
    Let's see an example of one.

    \begin{definition}[Topos of trees]
        Consider the following category called $\mathbb{N}_\rightarrow$:
        \begin{center}
            % https://tikzcd.yichuanshen.de/#N4Igdg9gJgpgziAXAbVABwnAlgFyxMJZABgBpiBdUkANwEMAbAVxiRGJAF9T1Nd9CKAIzkqtRizZCuPEBmx4CRAEyjq9Zq0QhlM3goFEAzGvGa2RvXL6LByACymNk7QDp3XMTCgBzeEVAAMwAnCABbJBMQHAgkZW4g0IjEVWjYxCEEkBDwpBE0pGIsnOTHAsQjTgpOIA
\begin{tikzcd}
    0 \arrow[r] & 1 \arrow[r] & 2 \arrow[r] & 3 \arrow[r] & ...
    \end{tikzcd}
        \end{center}
    The \emph{topos of trees} is the category of presheaves on $\mathbb{N}_\leftarrow$
    written $\mathsf{Psh}(\mathbb{N}_\rightarrow)$.
    \end{definition}
    
    \item We haven't defined what a topos is yet, so just treat the above as the 
    definition for a special kind of presheaf category. But, let's see what a subobject 
    classifier is in this category.
    First, let's look at what monomorphisms look like. Here is an example of a 
    monomorphism:

    \begin{center}
        \includegraphics[width=200px]{fig/topos-1.png}
    \end{center}

    The red arrow is a natural transformation $Q \Rightarrow P$ where each 
    component of the natural transformation is an injective function.
    In a situation analogous to the one in finite sets, a natural transformation 
    $\alpha : P \Rightarrow Q$ is a monomorphism in a presheaf category if and only 
    if each component is an injective map.

    \item Objects of this category can be thought of as ``sets of facts that are true 
    at particular times''. They will correspond to \emph{Kripke models}, and 
    a monomorphism will correspond to a Kripke sub-world.

    % \item Syntax of modal logic:
    % \begin{align*}
    %     \phi, \psi ::= \phi \land \psi \mid \phi \lor \psi \mid \neg \phi \mid \square \phi \mid \diamond \phi \mid x
    % \end{align*}

    % Intuition: $\diamond \varphi$ says ``there is an accessible world where $\varphi$ holds'',
    % and $\square \varphi$ says ``$\varphi$ always holds''.

    \item Now we can think about the object of truth values. It is this:

    \begin{center}
        \includegraphics[width=250px]{fig/truth-1.png}
    \end{center}

    Formally, $\Omega$ is a presheaf that maps each object to the 
    set  $\{\top, 1, 2, \cdots\}$, and each 
    morphism to the predecessor function that maps $1$ to $\top$ and $\top$ to $\top$.
    Intuitively, think of each element of each set as representing ``the number 
    of steps until something is true''. With this intuition, a non-zero value represents 
    that something is false at a particular time step.

    \item Given this object of truth values, we can design the subobject 
    classifier as taking each time step and mapping it to the number of 
    steps of ``gas'' that timestep has left:

    \begin{center}
        \includegraphics[width=250px]{fig/truth-2.png}
    \end{center}

    
    \item Let's use this object of truth values to find the classifying morphism for 
    $\alpha$ above. We need to find a morphism $P \mor{\chi_\alpha} \Omega$ 


    \item  The mere existence of subobject classifiers already forces your category 
    to behave in set-like ways. An example:
        Every category with a subobject classifier is \emph{balanced}, 
        meaning that if there is an epimorphism and monorphism between 
        two objects, they are isomorphic. This is just like in sets where 
        the existence of an injection and surjection implies an isomorphism.
    
\end{itemize}


\section{Generalized elements}

Important picture: the subobject classifier gives us an interpretation for set-builder.\marginnote{Think of $\varphi(x)$ 
as an open term with free variable of type $X$.}

\begin{center}
 % https://tikzcd.yichuanshen.de/#N4Igdg9gJgpgziAXAbVABwnAlgFyxMJZABgBpiBdUkANwEMAbAVxiRAB13gAPAAk6xheADX7sAtlihj6AJzQALLAApuASk4BfEJtLpMufIRRkAjFVqMWbYTr0gM2PASKnyF+s1aIQpu-qcjV1Jzak9rH04AeXEYAHM6HQsYKDj4IlAAM1kIcSQyEBwIJDdCuiwGNgUICABrfxBs3PzqIqQAJl0snLzEUrbEAGYwq28OdjlFFXUGpt721uKhka82HFlrTQpNIA
\begin{tikzcd}
    \{x \in X \mid \varphi(x)\} \arrow[d, hook] \arrow[r] & 1 \arrow[d, "true"] \\
    X \arrow[r, "\varphi(x)"]                             & \Omega             
    \end{tikzcd}
\end{center}

We can say \emph{an element of X satisfies $\varphi$} if we can find a 
morphism $x_0$ such that the following commutes:

\begin{center}
    % https://tikzcd.yichuanshen.de/#N4Igdg9gJgpgziAXAbVABwnAlgFyxMJZARgBoAGAXVJADcBDAGwFcYkQAdD4ADwAIuWMHwAaAjgFssUcQwBOaABZYAFDwCUXAL4gtpdJlz5CKMsWp0mrdiN36QGbHgJEATBQsMWbRCGJ2DJ2M3UnMaL2tfLgB5CRgAc3oAh0NnE2RyUM8rHz9dCxgoePgiUAAzOQgJJEyQHAgkMjr6LEZ2RQgIAGtkiqqamnqkVz1yyurEJqHEAGZwnPYueSVVDV7x4cGG2fnvdhw5a1GQPomAFi2a49OkC7rtpojcngB9cnytIA
\begin{tikzcd}
    & \{x \in X \mid \varphi(x)\} \arrow[d, hook] \arrow[r] & 1 \arrow[d, "true"] \\
1 \arrow[ru] \arrow[r, "x_0"] & X \arrow[r, "\varphi(x)"]                             & \Omega             
\end{tikzcd}
\end{center}

This is often too strong, e.g. Kripke semantics. Enter generalized elements:


We say $U \vDash \varphi$ (read ``$U$ forces $\varphi$'').


% \section{The category of parts of objects}
% \begin{definition}[Slice category]
% Let $\calC$ be a category 
% \end{definition}

\section{Kripke-Joyal Semantics}
% https://ncatlab.org/nlab/show/Kripke-Joyal+semantics
\begin{itemize}
    \item The Kripke-Joyal semantics gives us a full picture of working internal 
    to a topos
    \item These will look very familiar if you are familiar with separation logic 
    papers
    % \item Let $\calE$ be an elementary topos and $\alpha : U \to X$ be a generalized element 
% of $x \in \calE$.

\end{itemize}

\section{Elementary topos}
\begin{itemize}
    \item The Kripke-Joyal semantics tells us what features we need our 
    category to support, which will be an elementary topos:
    
\end{itemize}

\begin{definition}[Elementary topos]
    A category $\calC$ is an elementary topos if and only if
    it has products, coproducts, exponentials, an initial object, a terminal object, 
    an object of truth values, and satisfies that for every object $X$ 
    the slice category $\calC / X$ has products.
\end{definition}
\begin{itemize}
    \item Every presheaf category is an elementary topos (!!)
\end{itemize}

%%%%%%%%%%%%%%%%%%%%%%%%%%%%%%%%%%%%%%%%%%%%%%%%%%%%%%%%%%%%%%%%%%%%%%%%%%%%%%%%
\chapter{Sheaf semantics}
\marginnote{This chapter is heavily based on these notes from 
Alex Simpson: \url{https://alexksimpson.github.io/Talks/TutorialOnSheafSemantics.pdf}}

Last time we mentioned that sheaves serve as a ``hash-lang for math''.
More precisely, this means that sheaves provide a model of
\emph{higher-order intuitionistic logic}.
In this logic it is possible to express a large amount of math;
as a rough approximation, think of what can be written down
as using Prop in Rocq (and without Inductives), or what can be expressed in Isabelle/HOL.
While the standard model of this logic interprets propositions as classical truth values (either true or false),
each category of sheaves provides a non-standard interpretation which often turns out to be convenient for modeling
situations that arise in PL.
This chapter provides a pared down presentation of the key idea behind how sheaves accomplish this.
Rather than covering full higher-order intuitionistic logic,
we will cover the intuitionistic propositional logic:
\[
    \varphi, \psi ::= \top \mid \bot \mid \varphi \land \psi
    \mid \varphi \lor \psi \mid \varphi \Rightarrow \psi \mid a
\]
This logic consists of the usual intuitionistic logical connectives, plus
some set of atomic propositions \(a\).

Eventually we will see how to interpret this logic using the so-called \emph{sheaf semantics}.
We will build up to it by first talking about Kripke semantics, then presheaf
semantics.

\section{Kripke semantics}

Kripke semantics are concerned with describing
    beliefs over time.
For example, in the past,
we believed that in the night sky there 
were 3 objects: a morning star and an
evening star.
Then, recently, we learned that in fact 
the morning star and evening star were 
both venus, and discovered new planets
like Saturn and Pluto.
\begin{center}
\includegraphics[width=200px]{fig/planet.png}
\end{center}
We can describe this situation by interpreting the connectives of intuitionistic propositional
logic with a generalized set of truth values.
The key idea of Kripke semantics is to introduce a set \(W\)
of \emph{possible worlds}, whose elements represent states of knowledge.
These states of knowledge are partially ordered according to what an \emph{accessibility relation} \(\le\);
the relation \(w \le w'\) (pronounced ``world \(w'\) is accessible from \(w\)'')
represents the situation where a state of knowledge \(w\) can evolve into \(w'\).
For the planetary example above, one could set \(W\)
to the three-element set \(\{\text{before},\text{recently},\text{now}\}\)
ordered by \(\text{before} \le \text{recently} \le \text{now}\).

Each proposition \(\varphi\) is interpreted as a subset of \(W\) giving the set of possible worlds in which \(\varphi\) holds.
Crucial to this interpretation is a \emph{monotonicity} property that models the fact that
knowledge is stable over time: if \(\varphi\) holds in world \(w\), then it also holds in all worlds accessible from \(w\).
The Kripke semantics of intuitionistic propositional logic is as follows.
\begin{align*}
  \llbr{-} &: \text{Formula} \to \wp(W) \\
  \llbr{\top} &= W \\
  \llbr{\bot} &= \varnothing \\
  \llbr{\varphi\land\psi} &= \llbr{\varphi} \cap \llbr{\psi}\\
  \llbr{\varphi\lor \psi} &= \llbr{\varphi} \cup \llbr{\psi}\\
  \llbr{\varphi\Rightarrow\psi} &=
  \{ w \mid \text{for all }w' \ge w\text{, if } w'\in\llbr{\varphi} \text{ then }w'\in\llbr{\psi}\}
\end{align*}
Of these definitions, the interpretation of \(\Rightarrow\) is the most interesting.
A naive interpretation that says \(\varphi\Rightarrow \psi\) holds in world \(w\)
if and only if \(w\in\llbr{\varphi}\) implies \(w\in\llbr{\psi}\)
is not valid here, because it fails to satisfy monotonicity.

These semantics are designed so that you have sound interpretations
of the proof rules of intuitionistic propositional logic: if $\varphi_1, \cdots, \varphi_n \vdash \psi$
then $\llbr{\varphi_1} \cap \cdots \cap \llbr{\varphi_n} \subseteq \llbr{\psi}$,
where $\vdash$ is intuitionistic syntactic entailment.
In this sense the Kripke semantics provides a ``domain-specific reinterpretation''
of the usual rules of propositional logic, where each formula \(\varphi\)
denotes a monotone subset \(\llbr{\varphi}\) of the poset \(W\).

As an example of this kind of reinterpretation,
consider the poset \(\N\) ordered by \emph{reverse} inclusion,
so that a natural number \(n\) is accessible from \(m\) if \(n\) is \emph{smaller} than \(m\).
This gives a semantics of intuitionistic propositional logic that is used in \emph{step-indexing}~\cite{dreyer2009logical}.

\section{Presheaf semantics}

Presheaf semantics generalizes the poset \((W,\le)\) from Kripke semantics
to a category \(\calC\).
Elements \(w\in W\) are generalized to objects \(c\) of \(\calC\)
and the accessibility relation \(w' \ge w\) is generalized to a morphism \(f : c' \to c\).
There are two things to note about this generalization.
First, there can now be multiple such morphisms \(f\) between any two given objects;
intuitively, this is like a generalization of traditional Kripke semantics
where there can be multiple different ``ways'' of going from one world to another.
Second, there is a bit of reversal of direction going on:
the accessibility relation \(w' \ge w\) becomes the existence of a morphism \(c' \to c\).
This reversal of direction has to do with the fact that presheaves on a category \(\calC\)
are contravariant; i.e., functors \(\calC^{\mathsf{op}}\to \Set\).

Rather than interpreting each formula as a monotone subset of \(W\),
the presheaf semantics interprets each formula as what is called a \emph{sieve} on \(\calC\).
\begin{definition}
  A \emph{sieve} on a category \(\calC\) is a collection of morphisms \(S\) in \(\calC\)
  that is closed under precomposition in the sense that if \(c' \mor{f} c\) is in \(S\)
  then so is \(c'' \mor{g} c' \mor{f} c\) for any morphism \(c'' \mor{g} c'\).
\end{definition}
Intuitively, a sieve \(S\) containing a morphism \(c' \mor{f} c\) represents
a proposition \(\varphi\) where, if one is currently in world \(c\),
then \(\varphi\) will become true if one transitions to world \(c'\) via \(f\).
The requirement that \(S\) be closed under precomposition
provides a generalization of monotonicity.

The \emph{presheaf semantics} is the corresponding generalization of Kripke semantics
where propositions are interpreted as sieves:
\begin{align*}
  \llbr{-} &: \text{Formula} \to \{\text{sieves on }\calC\} \\
  \llbr{\top} &= \text{the set of all morphisms in \(\calC\)} \\
  \llbr{\bot} &= \varnothing \\
  \llbr{\varphi\land\psi} &= \llbr{\varphi} \cap \llbr{\psi}\\
  \llbr{\varphi\lor \psi} &= \llbr{\varphi} \cup \llbr{\psi}\\
  \llbr{\varphi\Rightarrow\psi} &=
  \{c' \mor{f}c \mid
  \text{for all \(c'' \mor{g} c'\), if \(f\circ g \in \llbr{\varphi}\) then \(f\circ g \in \llbr{\psi}\)}
  \}
\end{align*}
Just like in the Kripke semantics, the presheaf
semantics for Kripke logic will also validate intuitionistic
logic entailments.

The following proposition verifies that the presheaf semantics indeed generalizes the posetal case.
\begin{proposition}
  For any poset \((W,\le)\), monotone subsets of \(W\)
  are the same as sieves on \(W^{\mathsf{op}}\), where \(W\)
  is considered as a thin category,
  and the presheaf semantics agrees with the Kripke semantics.
\end{proposition}
In particular, the Kripke semantics for step-indexed logic corresponds to
the presheaf semantics for the usual poset of natural numbers \((\N,\le)\) made into a thin category.
(Note that, unlike in the last section, this poset has the natural numbers ordered in the usual way.
The appearance of \(\mathsf{op}\) in the presheaf semantics automatically performs the reversal
of the ordering relation that we did manually in the last section.)

%% \begin{itemize}
%%     \item Now we can begin to generalize from the previous situation
%%     to be categorical.
%%     \item The partial order of worlds $(\mathbb{W}, \le)$ becomes
%%     a category $\calC$.
%%     \item Continuing with our example, we have $\calC$ is:

%%     \begin{center}
%%        % https://tikzcd.yichuanshen.de/#N4Igdg9gJgpgziAXAbVABwnAlgFyxMJZABgBpiBdUkANwEMAbAVxiRACEYAzCAJ1YC+pdJlz5CKAIzkqtRizYAlGAGMYYHAwCeIISOx4CRAEwzq9Zq0QgAchADuu2TCgBzeEVBdeEALZIyEBwIJEk9EG8-UOpgpGMBCgEgA
%% \begin{tikzcd}
%%     Before \arrow[r] & Recently \arrow[r] & Now
%%     \end{tikzcd}
%%     \end{center}

%%     \item A  monotone subset (i.e., a Kripke monotone predicate)
%%     is a presheaf $\calC^\text{op}$. For example,
%%     we can have the following presheaf:

%%     \begin{center}
%%         \includegraphics[width=300px]{fig/planet-psh.png}
%%     \end{center}

%% \end{itemize}

%% \subsection{Formulas as subpresheaves}

%% \begin{itemize}
%%     \item Now we want to be able to interpret formulae $\varphi$
%%     with some free variables where $\Gamma \vdash \varphi$.
%%     As usual, we can associate the context $\Gamma$ with some
%%     object $\lceil \Gamma \rceil$
%%     \item An entailment $\Gamma \vdash \varphi$ is a \emph{subpresheaf},
%%     which is a part of a presheaf that respects Kripke monotonicity.
%%     I.e., it is a monic natural transformation $U \to \lceil \Gamma \rceil$
%%     for some presheaf $U$

%%     \item For example, the following is a subpresheaf: \todo{}

%%     \item Now we can define a semantic interpretation $\lceil
%%     \rceil(w)$ as a collection of morphisms $w_1 \to w$
%%     that satisfies a category-theoretic analog of Kripke monotinicity
%%     called \emph{generalized monotinicity:}
%%     if $(w' \mor{f} w) \in \lceil \Gamma \vdash \varphi \rceil(w)$
%%     and $w'' \mor{g} w'$, then the composite
%%     $(w'' \mor{g} w \mor{f} w) \in \lceil \Gamma \vdash \varphi \rceil(w)$.

%%     \begin{align*}
%%         \lceil \top \rceil(w) &= \{ w' \mor{f} w \mid w' \text{ any object}\} \\
%%         \lceil \bot \rceil(w) &= \emptyset \\
%%         \lceil \varphi \land \psi \rceil &=
%%         \{f  : w' \to w \mid f \in \lceil \varphi \rceil (w)
%%         \text{ and }
%%         f \in \lceil \psi \rceil (w)\}\\
%%         \lceil \varphi \lor \psi \rceil &=
%%         \{f  : w' \to w \mid f \in \lceil \varphi \rceil (w)
%%         \text{ or }
%%         f \in \lceil \psi \rceil (w)\}
%%         \lceil \varphi \Rightarrow \psi \rceil &= \\
%%         \{f : w' \to w \mid \forall g : w'' \to w'.
%%         \text{if } (w''' \mor{g} w' \mor{f} w) \in \lceil \varphi \rceil&
%%         \text{ then } (w''' \mor{g} w' \mor{f} w) \in \lceil \psi \rceil\}
%%     \end{align*}
%%     \item These are the \emph{generalized Kripke semantics}:
%%     these apply to any presheaf.
%% %     \item Implication is interesting:  you interpret
%% %         $\lceil \varphi \Rightarrow \psi \rceil(w)$ as a monic
%% %         map of type
%% %         % https://tikzcd.yichuanshen.de/#N4Igdg9gJgpgziAXAbVABwnAlgFyxMJZABgBpiBdUkANwEMAbAVxiRAFUQBfU9TXfIRQBGclVqMWbADrSGAYxhYGAAlkBxOgFstdNdIBOi5d3EwoAc3hFQAMwMQtSMiBwQko13ROIvJrhRcQA
%% % \begin{tikzcd}
%% %     U \arrow[r, tail] & \lceil \Gamma \rceil
%% %     \end{tikzcd}
%% %     We can reverse engineer what this monic map is by examining
%% %     the order-theoretic definition:
%% %     \begin{align*}
%% %         \Big\{ w \mor{g} w_{cur} \mid \forall w' \mor{f} w . \text{if }
%% %         (w' \mor{f} w \mor{g} w_cur) \in \lceil \phi \rceil
%% %         \text{ then } (w' \mor{f} w \mor{g} w_cur) \in \lceil \psi \rceil \Big\}
%% %     \end{align*}

%% \end{itemize}

A more interesting example is the presheaf semantics for 
    name generation. 
We will take as our indexing category the category of 
    finite injective maps $\mathsf{FinInj}^{\mathsf{op}}$.
    Objects are finite sets $\Gamma$ and 
    homs are injection functions $\Gamma \to \Gamma'$.
Intuitively, this represents how names can change over time:
    an injective function $\Gamma \to \Gamma'$
    expands the set of available names from \(\Gamma\) to \(\Gamma'\),
    while possibly renaming the variables in \(\Gamma\) along the way.

In the resulting presheaf semantics, the interpretation \(\llbr{\varphi}\)
of a proposition \(\varphi\) is the set of renamings
$r : \Gamma \to \Gamma'$ such 
that $\varphi$ holds when $\Gamma'$ is ``available'', i.e., 
$\Gamma$ is rewritten into some new context $\Gamma'$ that 
validates $\varphi$. 

Notice that this is \emph{proof-relevant} in the sense 
that the kind of substitution performed matters: there can be 
many injective maps $\Gamma \to \Gamma'$.
As usual the interesting case here is implication,
which can be read off directly from the Kripke semantics above.

\section{Sheaves}
At this point we've hit most of the common PL applications.
    Presheaf semantics get you very far: it is not that common to
    need to use sheaves.

Whereas presheaf and Kripke semantics are well-suited to modeling
\emph{when} a proposition becomes true,
sheaves are well-suited to modeling \emph{where}
a proposition becomes true.
Sheaves interpret objects \(c\) of the indexing category \(\calC\)
not as states of knowledge, but rather as \emph{spaces};
morphisms \(f : c' \to c\) represent \emph{subspace inclusions}
that map a smaller space \(c'\) into a larger space \(c\).
The sheaf semantics then models the spaces where a given proposition is true.
The monotonicity requirements of Kripke and presheaf semantics
continue to be relevant under this spatial reading:
if a proposition holds in a space \(c\),
then one would intuitively still expect it to hold in a subspace \(c'\)
picked out by some inclusion \(f : c' \to c\).
But the spatial setting provided by sheaf semantics adds an additional twist
not present in the more temporal reading provided by presheaf and Kripke semantics:
it is possible for a family of ``subspaces'' \(\{f_i : c'_i \to c\}_{i\in I}\)
to jointly \emph{cover} a space \(c\).

The spatial interpretation offered by sheaf semantics affects the reading of disjunction.
The Kripke/presheaf reading of disjunction says that \(\varphi\lor\psi\)
holds on a space \(w\) if either \(\varphi\) holds on all of \(w\)
or \(\psi\) holds on all of \(w\). But there is a third possibility:
perhaps \(\varphi\) holds on some piece \(w_1\) of \(w\),
and \(\psi\) holds on some piece \(w_2\) of \(w\),
and together \(w_1,w_2\) cover \(w\).

A simple example is given by interpreting ``space'' as ``subset of \(\R\)''
and ``inclusion'' as subset inclusion.
These spaces along with their inclusion relations forms a thin category \(\wp(\R)\).
Families of morphisms with a common codomain \(U\) in this setting
are families \((U_i)_{i\in I}\) of subsets of \(\R\)
where \(U_i \subseteq U\) for all \(i\in I\).
It is then natural to say that such a family \emph{covers} \(U\)
if they union to \(U\): \(\bigcup_i U_i = U\).
The sheaf semantics for this notion of cover
interprets each proposition \(\varphi\)
as a collection \(\llbr{\varphi}\) of subsets of \(\R\);
a subset \(U\) is in \(\llbr{\varphi}\)
if ``\(\varphi\) is true everywhere in \(U\)''.
For example, one can define an atomic proposition \(\varphi =\)``is less than one''
whose interpretation \(\llbr{\varphi}\) is the collection of all those subsets \(U\) of \(\R\)
contained in the \((-\infty,1)\),
and an atomic proposition \(\psi =\)``is greater than zero''
whose interpretation \(\llbr{\psi}\) is the collection of all those subsets contained in \((0,\infty)\).
Now consider the disjunction \(\varphi \lor \psi\).
Intuitively, this proposition should hold \emph{everywhere} in \(\R\),
as every real number is either less than one or greater than zero:
we should expect \(\llbr{\varphi\lor\psi} = \llbr{\top} = \) the set of all subsets of \(\R\).
But the presheaf semantics says that \(\llbr{\varphi\lor\psi} = \llbr{\varphi} \cup \llbr{\psi}\),
which consists only of those subsets of \(\R\) are either contained in \((-\infty,1)\)
or \((0,\infty)\).
The sheaf semantics corrects this by taking into account the fact that \((0,\infty)\) and \((-\infty,1)\)
form a \emph{cover} of \(\R\): a disjunction \(\varphi\lor\psi\) holds on a subset \(U\) of \(\R\)
under the sheaf semantics if there is a covering \(U = U_1 \cup U_2\) of \(U\) by two subspaces \(U_1,U_2\)
such that \(\varphi\) holds on \(U_1\) and \(\psi\) holds on \(U_2\).

More generally,
the sheaf semantics is parameterized by a relation \(T\) called a \emph{coverage}
that says when a given family of morphisms \(\{f_i : c_i \to c\}_{i\in I}\) with common codomain \(c\)
counts as a cover of \(c\).
\begin{definition}
  A \emph{coverage} on a category \(\calC\) is a relation \(T\)
  between families of morphisms with common codomain \(c\) and objects \(c\) of \(\calC\).
  If \(T\) relates a family \(\{f_i : c_i \to c\}_{i\in I}\) to an object \(c\) we say that
  \(\{f_i : c_i \to c\}_{i\in I}\) is a \emph{\(T\)-cover} of \(c\).
  The relation \(T\) must satisfy the following properties:
  \begin{itemize}
    \item Reflexivity: the singleton set \(\{\mathsf{id}_c : c \to c\}\) is in \(T\) for every object \(c\)
    \item Transitivity: if \(\{f_i : c_i \to c\}_{i\in I}\) is in \(T\) and for every \(i\in I\)
      there is a family \(\{f_{ij} : c_{ij} \to c_i\}_{j\in J_i}\) in \(T\),
      then the family \(\{ f_i f_{ij} : c_{ij} \to c \}_{i\in I,j\in J_i}\) is in \(T\)
    \item Stability: if \(\{f_i : c_i \to c\}_{i\in I}\) is in \(T\)
      and \(g : c' \to c\) is an arbitrary morphism into \(c\),
      then there exists a family \(\{f_j' : c_j' \to c'\}_{j\in J}\)
      such that every \(gf_{j}'\) factors through some \(f_i\);
      that is, for every \(j\in J\) there exists an \(i\in I\)
      such that \(gf_{j} = f_ig'\) for some \(g'\).
  \end{itemize}
\end{definition}
These coverage conditions are best visualized as closure conditions on families of morphisms drawn as trees.
Reflexivity says that the trivial tree with just an identity morphism counts as a cover;
transitivity says that a family of covering trees can be grafted onto another to form a new covering tree;
stability says that a covering tree can be ``pulled back'' along a morphism \(g : c' \to c\).

With the definition of coverage in hand, the sheaf semantics of intuitionistic propositional logic
is as follows.
\begin{align*}
  \llbr{-} &: \text{Formula} \to \{\text{sieves on }\calC\} \\
  \llbr{\top} &= \text{the set of all morphisms in \(\calC\)} \\
  \llbr{\bot} &= \{f : c' \to c \mid \text{\(\varnothing\) is a \(T\)-cover of \(c\)}\}
  \\
  \llbr{\varphi\land\psi} &= \llbr{\varphi} \cap \llbr{\psi}\\
  \llbr{\varphi\lor \psi} &= \{ f : c' \to c \mid \text{there is a \(T\)-cover \(\{f_i : c_i \to c'\}_{i\in I}\) of \(c'\)
    such that \(ff_i\in\llbr{\varphi}\) or \(ff_i\in\llbr{\psi}\) for all \(i\)}\} \\
  \llbr{\varphi\Rightarrow\psi} &=
  \{c' \mor{f}c \mid
  \text{for all \(c'' \mor{g} c'\), if \(f\circ g \in \llbr{\varphi}\) then \(f\circ g \in \llbr{\psi}\)}
  \}
\end{align*}
Like the presheaf semantics, this semantics interprets each proposition as sieve.
But the sheaf semantics satisfies an additional invariant.
Each sieve produced by the sheaf semantics is what is known as \emph{\(T\)-closed};
we omit the formal definition here, but intuitively this says that if a proposition holds
for every component of a covering family, then it holds for the space covered.

Examples of coverages:
\begin{itemize}
  \item The notion of cover for subsets of \(\R\) forms a coverage on \((\wp(\R),\subseteq)\) considered as a thin category.
  \item One can place a coverage on the category \(\mathsf{Formula}\) of propositional formulae
    that says that a finite set \(\varphi_1,\dots,\varphi_n\) covers a formula \(\varphi\)
    if \(\varphi_1\lor\dots\lor\varphi_n \dashv\vdash \varphi\).
    Intuitively, this notion of coverage corresponds to the notion of cover one would get
    by intepreting formulas as sets of models.
  \item A more exotic notion of coverage, for the category \(\mathsf{FinInj}^{\mathsf{op}}\),
    is given by declaring any singleton set to be a cover (that is, \(\{r : \Gamma \to \Gamma'\}\)
    covers \(\Gamma\) for any \(\Gamma,\Gamma',r\).).
    This coverage corresponds to a category of sheaves known as the \emph{Schanuel topos},
    and produces a sheaf semantics that can be used to talk about name generation
    as modeled by \emph{nominal sets}~\cite{pitts2013nominal}.
\end{itemize}

\subsection{Where are the categories?}

One might have noticed that categories of presheaves or sheaves have not played a role
in any of the above. This is because we have restricted the discussion to propositional logic.
Moving from propositional to first-order logic requires interpreting formulas-in-context \(\Gamma\vdash\varphi\),
which in turn requires interpreting contexts \(\Gamma\) and upgrading the semantics of formulas
to include substitutions \(\gamma \in \llbr{\Gamma}\).
It will turn out that this upgrade requires interpreting each \(\llbr{\Gamma}\) as a presheaf (in the presheaf semantics)
or sheaf (in the sheaf semantics).

We briefly sketch how this interpretation works.
Recall the planetary picture from before.
Turning this picture on its side, we obtain
a balloon diagram lying over the category \(W^{\mathsf{op}}\),
where \(W\) is the poset \(\text{before}\le\text{recently}\le\text{now}\).
This balloon diagram defines a presheaf \(\mathsf{Sky}\) on \(W^{\mathsf{op}}\),
whose elements model our conception of objects in the night sky over time.
This presheaf can serve as the interpretation of a basic sort \(\mathtt{Sky}\) of a first-order intuitionistic logic.
Logical predicates \(x\ofty\mathtt{Sky} \vdash \varphi(x)\) then denotes subpresheaves:
for instance, the points in the diagram highlighted in yellow depict a subpresheaf
corresponding to the predicate \(\varphi\) that out those objects that
are considered planets through to the present day.
