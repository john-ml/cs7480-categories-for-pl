\mdfdefinestyle{exstyle}{%
  linecolor=black,linewidth=1pt,%
  frametitlerule=true,%
  frametitlefontcolor=black,
  frametitlebackgroundcolor=gray!80,
  innertopmargin=\topskip,
}
\mdtheorem[style=exstyle]{exercise}{Exercise}

\theoremstyle{definition}
\newtheorem{solution}[theorem]{Solution}



\chapter{Categorical Noninterference}
\marginnote{This chapter was written by Bex Golovanov.}

Noninterference is a property in formal security to establish the safety of private information with respect to public information. Essentially, a program/language/protocol satisfies noninterference if publicly observable operations do not depend on private information; behavior using private data has no distinguishable effect on public data. 

This property is very strong, and almost always too strong. One might be motivated to research different models for reasoning about noninterference, with the hope of obtaining a weakened (yet formal) version. 
As was mentioned at the beginning of this semester, the answer to understanding a problem is rarely to categorify it. But why not give it a try? 

In this chapter, we explore Sterling and Harper's approach to noninterference via a sheaf semantics in their paper titled \textit{Sheaf Semantics of Termination-Insensitive Noninterference}\footnote{We do not cover anything pertaining to termination, sensitivity or lack thereof. Our exploration is also localized to pages 6 and 7 of the paper, as well as one observation on page 12 which we prove at the end of this chapter.}. 

The category theory in this chapter is primarily sheaf-theoretic; the content is almost entirely self-contained. Sources and supplementary reading can be found in these course notes, \S I and \S II of \textit{Sheaves in Geometry and Logic} by MacLane and Moerdijk, \textit{Category Theory in Context} by Emily Riehl, and of course ncatlab.org.
\vspace{1cm}

\section{Information Flow Basics}

This section introduces some terminology and basic ideas around formal reasoning for information flow. We start with the notion of a \textit{security lattice}. The simplest form is the set $\calL = \{L, H \}$ with $L \leq H$, where $L$ represents low-security (or public) data, and $H$ represents high-security (or secret) data. 

 One way to think about this lattice with respect to programming is by "labelling" variables with some $\ell \in \mathcal{L}$, e.g. $x_L$ represents that $x$ is a low-security variable. This labelling can be implemented with \textit{security types}. For example, if we want to talk about $x_L$ of type Int with this framework, we would say $x$ has type $Int_L$, where $Int_L$ represents public ints.
 

When working with information flow, we ideally want use of public (observable) data to not reveal anything about secret data. The strongest statement of this is requiring that public data never depends on secret data, usually referred to as noninterference. 


\begin{definition}[$\ell-$equivalence]
Let $L:\text{Var} \to \calL$ be a \textit{labelling} from variables $x$ to elements $\ell$ of the security lattice $\calL$. 

A state $\sigma:\text{Var} \to \text{Val}$ maps variables to values. Two states $\sigma,\sigma'$ are \textit{$\ell$-equivalent}, denoted $\sigma \sim_\ell \sigma'$, when for all $\ell'\leq \ell$, if $L(x) = \ell'$, then $\sigma(x) = \sigma'(x)$. This requirement is equivalent to $L(x) \leq \ell$ for all $x$, which we can interpret as $x$ being "public with respect to $\ell$".

\end{definition}

\begin{definition}[Noninterference]
    A program $\alpha$ satisfies \textit{noninterference} if, for any states $\sigma_1,\sigma_2$ such \\that $\sigma_1 \sim_L \sigma_2$, and $\sigma_1 \xrightarrow{\alpha} \sigma_1'$, 
    $\sigma_2 \xrightarrow{\alpha} \sigma_2'$, we have $\sigma'_1 \sim_L \sigma'_2$.
    If initial states agree on all public values, the respective post-states \\should agree on all public values.
\end{definition}
Let \textbf{Rel} be the category whose objects are sets and whose morphisms $A \sim_R B$ are relations $\sim_R \subseteq A \times B$. We can think of $\ell$-equivalence as a morphism in \textbf{Rel}, where $\sigma \sim_\ell \sigma'$ means $(\sigma,\sigma') \in\;\sim_\ell\;\subseteq \Sigma \times \Sigma'$ for $\Sigma,\Sigma'$ sets of states. Since functions $f:A \to B$ can be represented as graphs $f\;\subseteq A\times B$, where $(a,b) \in f$ if $f(a) = b$, we can consider functions to be relations as well, so $\alpha:\Sigma \to \Sigma'$ is also a morphism in \textbf{Rel}. 
    
Then noninterference can be understood as the requirement that the following diagram commutes:
    \[\begin{tikzpicture}
  \node (s1) at (0,2) {$\sigma_1$};
  \node (s2) at (3,2) {$\sigma_2$};
  \node (s1') at (0,0) {$\sigma'_1$};
  \node (s2') at (3,0) {$\sigma'_2$};
  
  \draw[<->] (s1) -- (s2) node[midway,above] {$\sim_\ell$};
  \draw[<->] (s1') -- (s2') node[midway,above] {$\sim_\ell$};
  
  \draw[->] (s1) -- (s1') node[midway,left] {$\alpha$};
  \draw[->] (s2) -- (s2') node[midway,right] {$\alpha$};
\end{tikzpicture}\]


    This property is also referred to as $\sigma_1$ and $\sigma_2$ being \textit{indistinguishable} with respect to public values.

This indistinguishability property is equivalent to requiring any map from some type $\tau_H$ to type $\tau_L$
to be constant. %\textit{weakly constant}. A function $f: X \to Y$ is weakly constant if for any $x_1,x_2 \in X$, $f(x_1) = f(x_2)$.
One characterization of this definition is the following:
\begin{proposition}
    For any closed function $\cdot \vdash f : \tau_H \to \tau_L$, there exists a closed $\cdot \vdash v : \tau_L$ such that $f \simeq \lambda\_.v$,
    meaning $f$ is observationally (extensionally) equivalent to a constant $v$ function.
\end{proposition}
\section{Categorical Background Potpourri}
In this section, we explain the medley of categorical concepts needed to understand Sterling and Harper's construction. %expand?
\subsection{Lattices and Logic}
\begin{definition}[Lattices]
\sloppy
    A lattice $\mathcal{L}$ is a partially ordered set which has all \textit{meets} and \textit{joins}.
    \begin{itemize}
        \item Two elements $l,k \in \mathcal{L}$ have a \textit{meet} if there exists an $m\in\mathcal{L}$
            such that $m$ is the greatest lower bound of $l$ and $k$. \textit{Joins} are symmetrically least upper bounds.
        \item Having all meets and joins means for any pair of elements in the lattice, you can find a greatest lower bound as well as a least upper bound. This property can also be phrased as having all binary products and all binary coproducts. 
    \end{itemize}
    A lattice with $\mathbf{0}$ and $\mathbf{1}$ (unique initial and terminal objects) has all finite limits and all finite colimits. 
\end{definition}
%People generally refer to a lattice of security levels $\ell$ when reasoning about information flow. The simplest example is the set $\{L,H\}$, where $L < H$, referring to low-security (public) and high-security (private) respectively.

Intuitionistic propositional logic can be thought of as a \textit{Heyting algebra }$H$, which is a lattice with $\mathbf{0}$ and $\mathbf{1}$, and has for each pair $x,y$ an exponential object $y^x$, usually written as $x \Rightarrow y$. Disjunction and conjunction are joins and meets, respectively, with the initial object $\mathbf{0}$ and terminal object $\mathbf{1}$ as their respective identities. That is,
\[ x \lor \mathbf{0} = x, \quad x \land \mathbf{1} = x\]

A \textit{distributive lattice} is a lattice in which 
\[x \land (y \lor z) = (x \land y) \lor (x \land z)\]
holds for all $x,y,z \in \calL$. This identity also implies the dual, $x \lor (y \land z) = (x \lor y) \land (x \lor z)$.

%We can also define negation, in terms of the notion of \textit{complement} for an element $x\in\calL$. In a distributive lattice, the complement is unique. If it exists, the unique complement of $x$, denoted $\neg x$, satisfies the following formulas:
% \[ x \lor \neg x = \mathbf{1}, \quad x \land \neg x = \mathbf{0}.\]

The exponential $x \Rightarrow y$ corresponds to implication. Writing out our usual diagram in the above language,
\[
\begin{tikzcd}[row sep=large, column sep=large]
z \land x \arrow[dr] \arrow[d] & \\
y^x \land x \arrow[r] & y
\end{tikzcd}
\]
we can characterize $y^x$ or $x \Rightarrow y$ by the following: 
\[z \land x \leq y \quad \iff \quad z \leq x \Rightarrow y\]
We can interpret $x \Rightarrow y$ as a least upper bound for $z$ such that $z \land x \leq y$. Negation in a Heyting algebra can be defined in the "implies absurdity" manner:
\[\neg x := x \Rightarrow \mathbf{0}\]
With our definition of $\Rightarrow$, this implies \[y \leq \neg x \quad \iff \quad y \land x = \mathbf{0},\]
where we have $=$ instead of $\leq$ because $\mathbf{0}$ is initial.

\marginnote{An alternative (cursed) answer to the question "what is a topology?" : a topology is \textit{just} something you define sheaves over, and sheaves are \textit{just} something you define a cohomology on, so the question you should really be asking is, what is a cohomology?}
\begin{definition}[Topology]
\sloppy
    A \textit{topology} on a set $X$ can be defined as a collection $\mathcal{T}$ of subsets of X, called \textbf{open sets} and 
    satisfying the following axioms:
    \begin{enumerate}
        \item The empty set and $X$ itself belong to $\mathcal{T}$.
        \item Any arbitrary union of members of $\mathcal{T}$ belongs to $\mathcal{T}$.
        \item The intersection of any finite number of members of $\mathcal{T}$ belong to $\mathcal{T}$.
    \end{enumerate}
\end{definition}
For a topological space $X$, the set of all open sets of $X$, notated $\calO(X)$, forms a Heyting algebra. This implies we can model the intuitionistic propositional calculus with any topological space $X$, by looking at the associated set $\calO(X)$. 
%potentially include Formula relation
\newpage
\begin{definition}[Special Subsets of Lattices]
\sloppy
    \begin{itemize}
        \item An \textit{upper set} U on a poset $(\calP,\leq)$ is upwards closed: if $x\in U$ and $x \leq z$, then $z \in U$. 
        \item Symmetrically, a \textit{lower set} U on $\mathcal{P}$ is downwards closed: if $x \in U$ and $z\leq x$, then $z\in U$.
        \item A \textit{filter} $\calF$ on $\calP$ is an upper set with the additional condition of every finite subset of $\calF$ having a lower bound in $\calF$ (filters are closed under finite meets)
    \end{itemize}
\end{definition}
\marginnote{For an example of an upper set which is \textit{not} a filter, consider the lattice of divisors of $12$ ordered by division. The upper set defined by $U = \{4,6,12\}$ is not a filter, since the meet (in this case gcd) of $4$ and $6$ is $2$, which is not contained in $U$. By contrast, $\{6,12\}$ satisfies filter requirements}
%\newpage
\begin{example}
Recalling that the natural numbers $\mathbb{N}$ form a poset ordered under divisibility, here is a Hasse diagram of the factors of $60$,
where the blue numbers are the elements of the upper set generated by $6$ (and in this case the upper set is also a filter), 
and the orange are elements of the lower set generated by $10$.

\begin{tikzpicture}[scale = 0.65]
  % Level 0: 1
  \node[orange] (1) at (0,0) {$\mathbf{1}$};
  
  % Level 1: prime divisors of 60
  \node[orange] (2) at (-2,2) {$\mathbf{2}$};
  \node (3) at (0,2) {$3$};
  \node[orange] (5) at (2,2) {$\mathbf{5}$};
  
  % Level 2: products of two primes
  \node (4) at (-3,4) {$4$};
  \node[cyan] (6) at (-1,4) {$\mathbf{6}$};
  \node[orange] (10) at (1,4) {$\mathbf{10}$};
  \node (15) at (3,4) {$15$};
  
  % Level 3: products of three primes
  \node[cyan] (12) at (-2,6) {$\mathbf{12}$};
  \node (20) at (0,6) {$20$};
  \node[cyan] (30) at (2,6) {$\mathbf{30}$};
  
  % Level 4: 60
  \node[cyan] (60) at (0,8) {$\mathbf{60}$};
  
  % Edges from 1
  \draw (1) -- (2);
  \draw (1) -- (3);
  \draw (1) -- (5);
  
  % Edges from 2
  \draw (2) -- (4);
  \draw (2) -- (6);
  \draw (2) -- (10);
  
  % Edges from 3
  \draw (3) -- (6);
  \draw (3) -- (15);
  
  % Edges from 5
  \draw (5) -- (10);
  \draw (5) -- (15);
  
  % Edges from 4
  \draw (4) -- (12);
  \draw (4) -- (20);
  
  % Edges from 6
  \draw (6) -- (12);
  \draw (6) -- (30);
  
  % Edges from 10
  \draw (10) -- (20);
  \draw (10) -- (30);
  
  % Edges from 15
  \draw (15) -- (30);
  
  % Edges from 12
  \draw (12) -- (60);
  
  % Edges from 20
  \draw (20) -- (60);
  
  % Edges from 30
  \draw (30) -- (60);
  
\end{tikzpicture} 
\end{example}
\begin{definition}[Alexandroff Topology]
\sloppy
    The \textit{Alexandroff Topology} is a natural topology on the underlying set of a pre-ordered set, where the open sets are the upper sets.
\end{definition}
\marginnote{(Foreshadowing): If we order filters $\calF$ on a poset $\calP$ by inclusion, we get a poset of the set of filters. This would imply there exists an Alexandroff topology to be defined here, with upper sets of the filters. }
Let's check that this definition satisfies the axioms for open sets.
\begin{proposition}
    Upper sets of a poset $\mathcal{P}$ satisfy the following axioms:
    \begin{enumerate}
        \item The empty set and $\mathcal{P}$ itself are upper sets.
        \item Upper sets are closed under arbitrary unions.
        \item Upper sets are closed under finite intersections.
    \end{enumerate}
\end{proposition}

 \begin{proof}.
    \newline
        \begin{enumerate}
            \item $\emptyset$ as an upper set is vacuously true; there are no elements to check. For $\mathcal{P}$,
        suppose $x \in \mathcal{P}$ and $x\leq y$. Well $y\in\mathcal{P}$ because $\mathcal{P}$ is the entire set.
            \item Let $\{U_i\}_{i \in I}$ be a collection of upper sets. We wish to show $\bigcup_{i \in I} U_i$ is an upper set.
            Suppose $x \in \bigcup_{i \in I} U_i$ and $x \leq y$. Then $x \in U_j$ for some $j \in I$. $U_j$ is an upper set
            and $x\leq y$, so $y \in U_j$. Then $y \in \bigcup_{i \in I}U_i$ so arbitrary union of upper sets is also an upper set.
            \item Let $U,V$ be upper sets. We want to show $U \cap V$ is an upper set. Suppose $x\in U \cap V$ and $x\leq y$.
            Then $x \in U$ and $x \in V$. Since $U$ and $V$ are both upper sets and $x\leq y$, we have $y \in U$ and $y \in V$. Then $y \in U\cap V$!
            So upper sets are also closed under finite intersection.  
        \end{enumerate}
\end{proof}


%\newpage
\subsection{Sheaves}

In the sheaf semantics chapter we defined a \textit{coverage}, which was defined to be a relation $T$ between families of morphisms with common codomain $c$ and objects $c$ of $\calC$. If a family $\{f_i: c_i \to c\}$ and object $c$ are in $T$, then $\{f_i : c_i \to c\}$ is a $T$-cover of $c$. Conditions for $T$ can be found in the aforementioned chapter. A general intuition for these families of morphisms is to think of them as a way to partition an object $c$ by images of $f_i$. 
\newpage
\begin{definition}[What is a sheaf, actually?]
%For this explainer, we refer to sheaves over topological spaces. Given a space $X$, a sheaf is a functor $\calO(X)^{op} \to$ \textbf{Set}, meaning sheaves act on the open sets of $X$. 

Let $\calC$ be a category with a coverage $T$. For any object $A \in \calC$, we have a T-cover $\{a_i : A_i \to A\}_{i \in I}$ indexed by some $I$. A presheaf $F$
on $\calC$ is a sheaf if it satisfies two additional conditions for \textit{any}\\T-cover:
\begin{enumerate}
    \item[1.] \textit{locality}: For any $f,g \in F(A)$, if $f\mid_{A_i} = g\mid_{A_i}$ for all $i \in I$, then $f = g$. The notation $f\mid_{A_i}$ is equivalent to applying $F(p_i)(f)$ \\where $p_i$ is the morphism $A_i \to A$.
    
    If two elements agree locally 
    (on each part of the cover), they \\agree everywhere. We can also think of this as being able\\ to partition any $f \in F(A)$ into the "sum of its parts", where 
    the \; parts correspond to an arbitrary cover of $A$. 
    \item[2.] \textit{gluing}: Let $f \in F(A)$. 
    For any $i,j \in I$, we have $F(\pi_i)(f\mid_{A_i}) = F(\pi_j)(f\mid_{A_j})$  in $F(A_i \times_A A_j)$, where $\pi_i$ is the projection\\ $A_i \times A_j \to A_i$, $\pi_j$ is the projection for $A_j$ and $A_i \times_A A_j$ is the pullback of $A$ along $f\mid_{A_i}$ and $f\mid_{A_j}$.  
    
    If $\calC$ is a preorder with meets, this simplifies to for any $x \in A_i \cap A_j$, we have $f\mid_{A_i} x = f\mid_{A_j} x$. 
    \begin{itemize}
        \item Consider the case where $F(A)$ maps to sets of functions. In \\this case, an equivalent definition to the above which may\\ have more of a "gluing" connotation says that, for any \\$f_i : A_i \to A$ and $f_j : A_j \to A$ such that $f_i(a) = f_j(a)$ for $a \in A_i \cap A_j$, there exists a unique $f$ such that $f\mid_i = f_i$ and $f\mid_j = f_j$. That is, when we have two maps agreeing at their intersection, we can "glue" them together to get another map.
    \end{itemize}
\end{enumerate}
\end{definition}

\begin{proposition}
\label{Proposition:eq->sheaf}
    Let $\calC$ be a category with a coverage and $F : \calC^{op} \to Set$. For any $A \in \calC$, for any cover $\{a_i : A_i \to A\}$ of $A$, $F$ is a sheaf iff $F$ satisfies 
    the equalizer diagram 
  \[F(A) \xlongrightarrow{e} \prod_{i \in I} F(A_i) 
    \begin{matrix}
    \xlongrightarrow{p} \\
    \xlongrightarrow{q}
    \end{matrix}
    \prod_{i,j \in I} F(A_i \cap A_j),\]
    where $p(f_i) = f_i \mid_{A_i \cap A_j}$ and $q(f_i) = f_j \mid_{A_i \cap A_j}$.
\end{proposition}

Brief sketch on why the above is true:
\begin{lemma}
\label{eqmonl}
    $e$ (or any equalizing map) is a monomorphism. Conversely, the morphism associated to the coequalizer is an epimorphism. These are not too hard to prove.

    Can see Solution \ref{eqmons}
\end{lemma}
\begin{itemize}
    \item Since $F$ maps to \textbf{Set}, $e$ is also injective. We can think of $e$ as the map that partitions some $f \in F(A)$,
    i.e. $e(f) = \{f\mid_{A_i}\}_{i \in I}$. Then agreeing locally, $f\mid_{A_i} = g\mid_{A_i}$, is equivalent to saying $e(f) = e(g)$. Injectivity of $e$ gives us
    $f = g$.
    \item The equalizer-ness of $F$ means $F(A)$ only consists of those $f$ where $p \circ e (f) = q \circ e (f).$ We are guaranteed 
    $\left( f \mid_{A_i} \right)\mid_{A_i \cap A_j} = \left( f \mid_{A_j} \right)\mid_{A_i \cap A_j}$, which is equivalent to saying $f\mid_{A_i} x = f\mid_{A_j} x$
    for all $x \in A_i \cap A_j$.
\end{itemize}

\begin{definition}[Topological Vocabulary]

    \begin{itemize}
        \item Let $U$ be an open subset of some $S$ with topology $\calT$. The family of open sets $C = \{U_i\}_{i \in i}$ indexed by some $I$ is an \textit{open cover} of $U$
    iff $\bigcup_{i \in I}U_i = U$.
        \item A family of open subsets $\{B_i\}_{i\in I}$ of $S$ form a \textit{basis} for $\calT$ iff for \\any $U\subset S$, there exists some $J \subset I$ such 
        that $\{B_i\}_{i \in J}$ is an open cover of $U$. That is, any element of $\calT$ (open set) can be \\represented as a union of some subset of the basis elements.
    \end{itemize}
\end{definition}
    For a space $S$, the topological definition of a covering corresponds to a coverage on $\calO(S)$, where the families of morphisms are inclusion maps. Recall $\calO(S)$ denotes the set of open sets of $S$, which form a poset (in fact, a heyting algebra) under inclusion. 

    Taking $C$ as an open cover of $U$ as above, we can write $C$ as \\$\{U_i \hookrightarrow U\}_{i \in I}$.

\begin{definition}[Subterminal Sheaf]
\sloppy
    Fix a category $\calC$
    \begin{itemize}
        \item Given a functor $F$, the functor $G$ is a \textit{subfunctor} of $F$ if $G(C)\subseteq F(C) \;\forall\; C\in\calC$. If $F$ is a sheaf on $\calC$,
        then $G$ is a \textit{subsheaf} if and only if for every open $U$ of $\calC$ and every element $f \in FU$, and every open covering $U = \bigcup U_i$, one has
        $f \in GU$ if and only if $f\mid_{U_i} \in GU_i$ for all $i$.
        \item The terminal sheaf is the terminal object $\mathbf{1}$ of $Sh(\calC)$. A \textit{subterminal sheaf} $F$ is a subsheaf of $\mathbf{1}$. $F$ is a subfunctor, meaning \newline $FC \subseteq \mathbf{1}C = 1$ for all $C\in \calC$. This means $FC$ can only be either $1$ or $\emptyset$.      
    \end{itemize}
\end{definition}
\begin{proposition}
    A \textit{subobject} of a sheaf $F$ in the category $Sh(\calC)$ is isomorphic to a subsheaf of $F$.
    \begin{proof}
        See \S II.3 in \textit{Sheaves in Geometry and Logic} by MacLane and Moerdijk.
    \end{proof}
    %\textcolor{red}{TODO}
\end{proposition}

\begin{theorem}
\label{Theorem:opens=sheaf}
    For any space $X$, there is an isomorphism \[\calO(X) \cong Sub_{Sh(X)}(\mathbf{1})\] of partially ordered sets.

    \begin{proof}$\Rightarrow$

        Given any open set $W$ of $X$, define a functor $S_W$ on open sets $U$ by $S_W(U) = 1$ if $U\subseteq W$ and $\emptyset$ otherwise. Recall $\mathbf{1}U = 1$ for all open $U$. Let $\{V_i\}$ be an arbitrary covering of $U$. If $S_W(V_i) = 1$ for all $i$, then $\bigcup V_i \subseteq W$, meaning $U \subseteq W$, so $S_W(U) = 1$. In the other direction, if $S_W(U) = 1$, all $V_i \subseteq U$ so $S_W(V_i) = 1$ for all $i$. So, we can see $S_W$ is a subsheaf of $\mathbf{1}$.
        
        $\Leftarrow$

        Let $S$ be a subsheaf of $\mathbf{1}$. Each $S(U)$ must then be either $1$ or $\emptyset$. If $S(U) = 1$ and $V \subseteq U$, and $S$ requires a morphism $S(U) \to S(V)$ meaning $1 \to S(V)$, it follows that $S(V)$ is nonempty and thus $S(V) = 1$. By the equalizer condition, if $\{V_i\}$ is an open cover of $U$ and $SU_i = 1$ for all $i$, then $SU = 1$. 

        Define $W = \bigcup\{U \in \calO(X) | SU = 1\}$. Then by equalizer condition we have $SU = 1$ if and only if $U \subseteq W$. Then we have $S = S_W$, so we have the bijection $W \iff S_W$.
    \end{proof}

\end{theorem}

\subsection{Special Sheaf Functors}
\label{ssf}
The crux of the Sterling and Harper's information flow model is based on the following monad definition:

"For any subspace $Q\subset P$, a sheaf $A \in Sh(P)$ can be restricted to $Q$, as in $A\mid_Q \in Sh(Q)$, and then extended again to $P$. This composite defines an \textit{idempotent} monad on $Sh(P)$ that we can interpret as removing any data from $P$ that cannot be seen from $Q$."

\marginnote{A monad $T$ is idempotent when $T^2 = T$ (i.e. $\mu = $iso).}

This construction is a specialization of a much more general property of sheaves, which we need to do some work to define.

\begin{definition}
    Let $X,Y$ be topological spaces. A function $f : X \to Y$ is \textit{continuous} if, for every open $V \subseteq Y$, the preimage $f^{-1}(V)$ is open in $X$.
\end{definition}
\begin{exercise}
\label{ex1e}
    All monotone maps in the Alexandroff topology are continuous.

    See Solution \ref{ex1s}
\end{exercise}

%A \textit{bundle} over an object $B$ is an object $E$ equipped with a morphism $E \xrightarrow{p} B$. All bundles over $B$ form a category: the slice category $\calC / B$.

\begin{definition}[Direct Image Sheaf Functor]
    
    Given a continuous $f$, each sheaf $F$ on $X$ yields a sheaf $f_*F$ on $Y$ \\defined for $V$ open in $Y$, by $(f_*F)V = F(f^{-1}V)$. Essentially, $f_*F$ is the composite functor 
   \[\calO(Y)^{op}\xlongrightarrow{f^{-1}}\calO(X)^{op}\xlongrightarrow{F}\textbf{Set}\]
   This sheaf is called the \textit{direct image} of $F$ under $f$. The map \\
   $f_*:Sh(X) \to Sh(Y)$ is a functor of sheaf categories.

\end{definition}
%\begin{remark}
   \marginnote{In the measure theory/qbs lecture, we briefly discussed the pushforward of a measure along a measurable map: $\mu_B(b) = \mu_A(f^{-1}(b))$ where $f^{-1}$ is the preimage. We can similarly think of $\mu_B$ as the composite functor $\Sigma_B \xrightarrow{f^{-1}} \Sigma_A \xrightarrow{\mu_A} [0,1]$, where we might want to denote $\mu_B$ by $f_*\mu_A$, and say $f_*$ is a functor of probability measures.}
%\end{remark}

A continuous $f: X \to Y$ also induces a functor $f^*:Sh(Y) \to Sh(X)$, where each sheaf $G$ on $Y$ yields a sheaf $f^*G$ on $X$. We typically can't define $f^*$ (the Inverse Image Functor) as simply as we were able to define $f_*$ because we don't know if $f(U)$ is open (continuity only gives us information about preimages). 

First we have the general and scary definition (though not the most general because we assume sheaves over topological spaces):
\begin{definition}[Inverse Image Sheaf Functor]
\label{Definition:invsheaf}
    Let $f:X \to Y$ be a continuous map of topological spaces. For any sheaf $G$ on $Y$, we define the \textit{inverse image} sheaf $f^*G$ on $X$ to be the \textit{sheafification} of the presheaf $U \mapsto \colim_{V \supseteq f(U)}G(U)$, where $U$ is any open set in $X$, and the limit is taken over all open sets $V$ of $Y$ containing $f(U)$. 
\end{definition}
Luckily for us, the only continuous maps we care about in this chapter are inclusion maps. Here is some sketchy / informal reasoning as to why $i^*F(U) = F(U)$ when $X \subseteq Y$ is open for open $U \subseteq X$.
\begin{itemize}
    \item When $X \subseteq Y$ is open any open $U \subseteq X$ is open in $Y$ as well.
    \item $F(U)$ would be the colimit of $F(V)$ where $V \supseteq i(U) = U$, because $U$ is clearly the greatest lower bound (limit) on sets containing $U$, so dually $F(U)$ would be the least upper bound (colimit) on $F(V)$ with $V \supseteq U$, since $F$ is contravariant.
    \item Then $i^*F(U) = F(U)$, and of course $F$ is a sheaf so $i^*F(U)$ automatically satisfies sheaf conditions for open $U$. 
\end{itemize} 

If $X$ is not open, we have a more complicated situation, where open $U \subseteq X$ is not guaranteed to be open in $Y$. 

If we are working in a topology $\calT$ where for every (not necessarily open) subset $V$ of a space $Y$ has a smallest open $U \subseteq Y$ such that $V \subseteq U$, then $i^*F(V) = F(U)$. Let's notate this smallest open $U$ for $V$ by $o(V)$. To have $i^*F$ satisfy sheaf conditions, we need covers to "behave well", that is, for $\bigcup_j V_j = V$, we need $\bigcup_j o(V_j) = o(V)$. In general this is not guaranteed. 

However, in this chapter we will only be looking at the Alexandroff topology. In this topology the open sets are upper sets, so for any set $V$, we have $o(V) =\uparrow V$, where $\uparrow V$ represents the upwards closure of $V$, basically turning $V$ into an upper set. The upwards closure $\uparrow$ respects/distributes across unions, that is $\bigcup_j \uparrow V_j = \uparrow \bigcup_j V_j$. So we do not need to worry about sheafifying anything!


%Luckily, for the purposes of this chapter, we only care about inclusion maps, which are \textit{open maps}, meaning the images of open subsets \textit{are} open! 

%\begin{definition}[Étale Spaces]
%    Let Top be a category of topological spaces
%    \begin{itemize}
%        \item A \textit{homeomorphism} is an isomorphism in Top (isomorphism of \quad \quad \; topological spaces)
%        \item Let $B$ be an object in Top. An \textit{etale space} over $B$ is a bundle \quad\quad\quad\;     
%        $p : E \to B$, such that for every $e \in E$, there exists an open subset $U$ containing $e$ such that $p(U)$ is open in $B$ and the restriction $p\mid_U: U \to p(U)$ is a homeomorphism.
%    \end{itemize}
%\end{definition}

%Essentially, étale spaces are defined precisely to give us information about openness in the image of a continuous map. For the purposes of this chapter, don't worry too much about them beyond that. 

%Let $f: X \to Y$ continuous in category $\calC$. Each étale bundle $p: E \to Y$, pulled back along $f$, yields a bundle $f^*E \to X$ over x, and $f^*$ is a functor $f^* : \textbf{Etale } Y \to \textbf{Etale } X$.  
%\[
%\begin{tikzcd}[row sep=large, column sep=large]
%f^*E \arrow[r] \arrow[d] & E \arrow[d, "p"] \\
%X \arrow[r, "f"'] & Y
%\end{tikzcd}
%\]

%We have an equivalence of categories $Sh(-) \cong \textbf{Etale }(-),$ with $\Lambda : Sh(-) \to \textbf{Etale }(-)$ and $\Gamma : \textbf{Etale }(-) \to Sh(-)$.


\begin{theorem}
    \label{Theorem:adjoint-sheaves}
If $f : X \to Y$ is a continuous map, then the functor $f^*$, sending each sheaf $G$ on $Y$ to its inverse image (preimage) on $X$, is left adjoint to the direct image functor $f_*$ : 
\[
\begin{tikzcd}
Sh(X) \arrow[r, shift left, "f_*"] & 
Sh(Y) \arrow[l, shift left, "f^*"]
\end{tikzcd}, \quad f^* \dashv f_*
\]
\begin{proof}
    See \S II.9 in \textit{Sheaves in Geometry and Logic} by MacLane and Moerdijk.
\end{proof}
\end{theorem}

\cref{Theorem:adjoint-sheaves} is included to note that $f_* \circ f^*$ is a monad on $Sh(Y)$. Particularly, when $X \subset Y$ and $f$ is the inclusion map $i$, we have a monad given by $i_* \circ i^*$. It is precisely this composition which defines Sterling et al.'s idempotent monad.

The preimage of the inclusion map $i : Q \hookrightarrow P$ is defined $i^{-1}(V) = V \cap Q$, where $V$ is an open subset of $P$. With this, let's compute this monad for any inclusion. Fix a sheaf $F$ and $V$ open in $P$.
\begin{align*}
    i_* (i^* F)(V) &= (i^* F)(i^{-1} V)\\
    &= (i^* F)(V \cap Q)
\end{align*}

Recall that in our special case, the functor $i^*$ operates by $i^*F(U) = F(\uparrow U)$ for open $U$ in subspace $Q$, where $\uparrow U$ is the upward closure of $U$ in $P$. When $Q$ is open in $P$, $i^*F(U) = F(U)$. 

For any sheaf $F$, $F(\emptyset) = 1$. We can see this from the equalizer definition, where $F$ in this case is the equalizer of the empty diagram, which is precisely the terminal object.

If $V \cap Q \neq \emptyset$, then $i_*(i^* F)(V) = F(\uparrow (V \cap Q))$. If $V\cap Q = \emptyset$, then $i_*(i^* F)(V) = 1$.

\subsection{Pushouts} 
Let $X$ be a topological space and $Y$ an open set in $X$. One of the monads that Sterling et al. define is the composition of $i_* \circ i ^*$ defined by $i : X\setminus Y \hookrightarrow X$. The authors mention that this monad $i_* \circ i^*$ can be identified with a pushout construction \[Y \sqcup_{Y \times (-)} (-): Sh(X) \to Sh(X),\] where $Y$ here is treated as the subterminal sheaf where $Y(U) = 1$ if $U \subseteq Y$, and $\emptyset$ otherwise. This construction is key for later noninterference proofs, so we provide some intuition for how pushouts work/behave.


Pushouts are dual to pullbacks, but it is not very intuitive to get a sense of what that means exactly. We can try to work it out by staring at the diagram in \textbf{Set}:

\[
\begin{tikzcd}[row sep=large, column sep=large]
A \arrow[r, "f"] \arrow[d, "g"'] & B \arrow[d, dashed, "i_B"] \arrow[ddr, bend left, "h"] & \\
C \arrow[r, dashed, "i_C"'] \arrow[drr, bend right, "k"'] & P \arrow[dr, dashed, "\exists! u"] & \\
& & X
\end{tikzcd}
\]

Where $P$ is the smallest object with morphisms $i_C, i_B$, such that $i_C\circ g = i_B\circ f$, i.e. $i_C(g(a)) = i_B(f(a))$ for all $a \in A$. Then we can see that this restriction is only for the images of $f$ and $g$; we don't have any equality condition on $b \notin im(f), c\notin im(g)$.  
These maps $i_B, i_C$ are canonically \textit{quotient maps} modulo the relation by $f$ and $g$, meaning they map $b \in B, c \in C$ to equivalence classes $[b],[c] \in P$, with this extra equivalence condition. 

So, we can interpret the commutativity of this square to mean that for $b,c$ where there exists an $a$ such that $f(a) = b, g(a) = c$, we require $i_B(b) = [b] = [c] = i_C(c)$.

In \textbf{Set}, $P$ is a "quotient of the disjoint union of $A$ and $B$", which can also be written $A\sqcup B/\sim$, where $\sim$ is the equivalence relation generated by the restriction detailed above, i.e. some $\sim$ satisfying $b \sim c$ when $\exists\; a\in A$ such that $f(a) = b, g(a) = c$. The pushout essentially collapses the images of $f$ and $g$ to equivalence classes in $P$. We provide a couple of examples in \textbf{Set}. 

\begin{example}[pushout of product]
    Let $B,C$ be nonempty sets.
\[
\begin{tikzcd}[row sep=large, column sep=large]
B \times C \arrow[r, "\pi_1"] \arrow[d, "\pi_2"'] & B \arrow[d, "i_B"] \\
C \arrow[r, "i_C"'] & B \sqcup_{B \times C} C \arrow[ul, phantom, very near start]
\end{tikzcd} \cong 
\begin{tikzcd}[row sep=large, column sep=large]
B \times C \arrow[r, "\pi_1"] \arrow[d, "\pi_2"'] & B \arrow[d, "i_B"] \\
C \arrow[r, "i_C"'] & \{*\} \arrow[ul, phantom, very near start]
\end{tikzcd}
\]
By definition of product, for every $b\in B$ and for every $c\in C$, there exists an $a \in A$, namely $(b,c)$, such that $\pi_1((b,c)) = b$ and $\pi_2((b,c)) = c$. Then every element of $B$ is in the same equivalence class as every element of $C$, and by transitivity of equality, every element of $B$ is in the same equivalence class, as well as every element of $C$. Then $P$ only has one equivalence class, so $P$ is a singleton set and thus isomorphic to the terminal object.

\end{example}
\begin{example}[pushout of the initial object, 0]
    Recall that if a category has a terminal object and pullbacks, then it has products. This is the dual for "initial and pushout imply coproducts".
    \[
\begin{tikzcd}[row sep=large, column sep=large]
0 \arrow[r, "!"] \arrow[d, "!"'] & B \arrow[d, "i_B"] \\
C \arrow[r, "i_C"'] & B \sqcup_{0} C \arrow[ul, phantom, very near start]
\end{tikzcd} \cong 
\begin{tikzcd}[row sep=large, column sep=large]
0 \arrow[r, "!"] \arrow[d, "!"'] & B \arrow[d, "i_B"] \\
C \arrow[r, "i_C"'] & B \sqcup C \arrow[ul, phantom, very near start]
\end{tikzcd}
\]
The empty set is the initial object in \textbf{Set}. There are no $b,c$ such that $f(a) = b, g(a) = c$ because there are no $a \in A \cong \emptyset$. Then each $b$ and $c$ are distinct, resulting in the disjoint union of $B$ and $C$. 

Going back to our product example, suppose $B$ is empty and $C$ is nonempty. The product $B \times C$ is isomorphic to $0$, so we would get $B\sqcup C = \emptyset\sqcup C = C$. If $B$ and $C$ are empty, then naturally the pushout would be empty as well.

\end{example}
Let's look at how pushouts behave with presheaves. Let $F,G \in [C^{op};Set]$.
\begin{example}[pushout of $F\times G$]
\label{Example:fxg}
    This example is interesting because it does not automatically collapse to $\{*\}$.
    Functors are defined by how they act on objects, so for some $A \in \calC$ we have 
    \[
    \begin{tikzcd}[row sep=large, column sep=large]
    (F \times G)(A) \arrow[r, "\pi_{1A}"] \arrow[d, "\pi_{2A}"'] & F(A) \arrow[d, dashed, "i_B"] \\
    G(A) \arrow[r, dashed, "i_C"'] & F(A) \sqcup_{(F\times G)(A)} G(A) \arrow[ul, phantom, very near start]
    \end{tikzcd} 
\]
Where the projection maps are now natural transformations, also defined component-wise. Recall $(F \times G)(A)\cong F(A)\times G(A)$. We end up with 3 (really 4) cases for the pushout:
\begin{enumerate}
    \item F(A) and G(A) are nonempty. Then we have the typical product - pushout, so in this case we do get $1$ or $\{*\}$.
    \item F(A) is nonempty and G(A) is empty. Then $F(A)\times G(A) = \emptyset$. Pushout from the initial object gives us $F(A) \sqcup G(A) = F(A) \sqcup \emptyset = F(A)$. We have a symmetric case for F(A) empty and G(A) nonempty.
    \item F(A) and G(A) are empty. This yields $\emptyset \sqcup \emptyset = \emptyset$
\end{enumerate}
Sheaves are a little more complicated. If we compute a pushout of a product of sheaves $S,P$ as above, we can define the result point-wise as a presheaf, but we are not guaranteed that this presheaf satisfies the sheaf conditions. \marginnote{The general solution to this is to \textit{sheafify} the computed presheaf pushout, which is currently out of the scope of this chapter.}
\end{example}

\newpage


%Fix a poset (partially ordered set) $\mathcal{P}$ of security levels closed under finite meets. For the remainder of this chapter we assume \[\mathcal{P} := \{L \subset M \subset H \subset \top\}.\]

%Sterling uses an Alexandroff topology, which is a structure of a topological space induced on the underlying set
%of a preordered set. 

\section{Sheaf Semantics for Noninterference (putting it all together)}
We are now ready to talk about some of the central ideas of the paper. This section will culminate in a proof of a form of noninterference where sheaves are interpreted as types. 

Fix a poset $\mathcal{P}$ of security levels closed under finite meets.

\begin{definition}
    \begin{itemize}
        \item An \textit{abstract behavior} $x$ is a filter on the poset $\mathcal{P}$. We can interpret this as saying $x$ denotes the security levels where a behavior is permitted.
        \begin{itemize}
            \item For example, a filter generated by M would include H and $\top$, capturing that medium-security information/behavior can be used at higher-security levels.
        \end{itemize}
        \item A \textit{security policy} U is a lower set in $\mathcal{P}$. The notation $U \vdash x$ is \\pronounced "U permits $x$", and means $U \cap x \neq \emptyset$. 
        \begin{itemize}
            \item Consider the poset $\{L \leq M \leq H\}$. Supposing we have a security policy generated by M: this includes L and M, and can be understood as denoting the security levels \textit{above} which \\some information/behavior is permitted. Alternatively, 
        the \\set where this behavior is not allowed.
        \end{itemize}
    \end{itemize}
\end{definition}
\begin{definition}
    The authors define $P$ to be the topological space where points are abstract behaviors, and the open sets are of the form $\{x \mid U \vdash x\}$.
\end{definition}

%\begin{proposition}
    %This space is the Alexandroff topology on the space of filters of $\calP$. 
    %\begin{proof}
        %The set of filters on $\calP$, ordered by inclusion, forms a poset. Once we show the sets $O_U := \{x \mid U \vdash x\}$ are upper sets of the filters, we are done.
        
        %Fix some lower set $U$ of $\calP$ and suppose $x \in O_U$. Let $y \in P$ such that $x \subseteq y$. Then $x \cap U \neq \emptyset$ immediately implies $y \cap U \neq \emptyset$ because $x \cap U \subseteq y \cap U$. 

        %Now, let $V$ be an upper set on Filters$(P)$ and suppose $x \in V$. 
   % \end{proof}
%\end{proposition}
%newpage
\begin{exercise}
    \label{ex2e}
    Prove these sets are open with the openness axioms.

    See Solution \ref{ex2s}
\end{exercise}


Each security level $\ell \in \mathcal{P}$ represents a security policy $\angled{\ell}$ of the form $\{a \in \calP \mid a \leq \ell\}$. The 
corresponding open subspace of $P$, $(\{x \mid \angled{\ell} \vdash x\})$ is denoted $P_{\angled{\ell}}$. The complement, $P \setminus P_{\angled{\ell}}$ is denoted $P_{\bullet \angled{\ell}}$.
\newpage
\begin{proposition}
\label{Proposition:ell-in-x}
    The predicate $\angled{\ell} \cap x \neq \emptyset$ is equivalent to $\ell \in x$. 
\end{proposition}
To see the above, suppose we have some nonempty intersection for $\angled{\ell}$ and $x$. Then there exists some $k \in \angled{\ell}$ such that $k \in x$.
Since $k \in \angled{\ell}$, this means $k \leq \ell$. Since $x$ is a filter and hence upwards closed, since $\ell \geq k$ and $k \in x$, we have $\ell \in x$. 
The other direction is immediate since $\ell \in \angled{\ell}$ so $\ell \in x$ clearly implies a nonempty intersection. 
%This construction is an Alexandroff topology on the space of filters on $\mathcal{P}$ --- the space where the points/elements are abstract behaviors.

\begin{proposition}
    Let $\angled{\ell_1}$ and $\angled{\ell_2}$ be security policies such that $\ell_1 < \ell_2$. Then
    $P_{\angled{\ell_1}} \subset P_{\angled{\ell_2}}$.
    \begin{proof}
        Let $x$ be an abstract behavior in $P_{\angled{\ell_1}}$, i.e. a filter on $\mathcal{P}$ such that ${\angled{\ell_1}} \cap x \neq \emptyset$.
        Then there is some $y$ in this intersection, meaning $y \in x$ and $y \leq \ell_1$. Since $\ell_1 < \ell_2$ and posets have transitivity,
        $y < \ell_2$, which means $y \in \angled{\ell_2}$. Then $y \in x \cap \angled{\ell_2}$, so $x \in P_{\angled{\ell_2}}$.
    \end{proof}
\end{proposition}

\subsection{Sheaf Interpretation}

In this subsection, we parse Sterling et al.'s statement: \textit{"our intention is to interpret each type of a dependency core calculus as a sheaf on the space $P$ of abstract behaviors. to see why this interpretation is plausible as a basis for secure information flow, we note that a sheaf on $P$ is the same thing as a presheaf on the poset $\mathcal{P}$"}. 

\begin{itemize}
    \item In this context, the only relevance of dependency core calculus is that it is a typed programming language where types can be annotated with security levels from some poset.
    \item This statement assumes a presheaf interpretation over $\calP$ is understood to be a plausible basis for modeling information flow, so let's justify this quickly:
    \begin{itemize}
        \item Consider the presheaf category over $\mathcal{P}$. This has a natural representation for information flow : for each security level $\ell$, we have a set $F(\ell)$ representing what data is visible under security level $\ell$. Since presheaves are contravariant functors, for each $k \leq l$ in $\calP$, we get a morphism $F(\ell) \to F(k)$, demonstrating the restriction of information to that at a lower security level.        
    \end{itemize}
    \item Now we have this note : a sheaf on $P$ is "the same thing" as a presheaf on the poset $\mathcal{P}$. This sounds like a bijection.
\end{itemize}


\begin{theorem}
\label{Theorem:P=S}
There is a bijection between the presheaf category $Psh(\calP)$ and the sheaf category $Sh(P)$.
\end{theorem}

Recall the topological space at hand, $P$, has open sets of the form $\{x \mid U \vdash x\}$, where $U$ is a lower set, $x$ is a filter, and 
$U \vdash x$ means $U \cap x \neq \emptyset$. Note that the open sets are defined by the lower sets of $\cal P$, so we will denote them by $O_U,$ where $U$ 
is the lower set of choice. Now we have a series of lemmas.

\begin{lemma}
    The family $\{P_{\angled{\ell}}\}_{\ell \in \calP}$ is a basis for the topology on $P$. 
    \begin{proof}
    %i think i almost have to tbh its so far apart at this point
    We know each $P_{\angled{\ell}} = \{x \mid \angled{\ell} \vdash x\}$ are open by definition. 

    Let $U$ be an upper set of $\calP$ and $O_U$ the corresponding open set of $P$, so 
    $O_U = \{x \mid x \cap U \neq \emptyset\} = \{x \mid \exists\; \ell\in U : \ell \in x\}$. We can tranform this into 
    \[\bigcup_{\ell \in U}\{x | \ell \in x\} = \bigcup_{\ell \in U}\{x \mid \angled{\ell} \vdash x\} = O_U,\]
    where the second expression comes from \cref{Proposition:ell-in-x}. So, we have a family of nonempty open sets of the form $P_{\angled{\ell}}$ indexed by some subset of 
    $\calP$ (as some of the sets in the above union will be naturally empty) such that their union is equal to $O_U$ in $P$. 
    \end{proof}
\end{lemma}
\begin{lemma}
    Let $\ell, k \in \calP$. We claim $P_{\angled{\ell}} \cap P_{\angled{k}} = P_{\angled{\ell \land k}}$
    \begin{proof}
        Note we have $\ell \land k$ for any $\ell,k \in \calP$ because $\calP$ is closed under finite meets. Suppose we have some $x \in P_{\angled{l\land k}}$,
        meaning $\ell \land k \in x$. Since $x$ is upwards closed and $\ell \land k$ is the meet of $\ell$ and $k$, it follows immediately that $\ell \in x$ and $k \in x$,
        meaning $x \in P_{\angled{\ell}} \cap P_{\angled{k}}$. 

        For the other direction, we begin with $\ell \in x$ and $k \in x$. Since $x$ is a filter, any finite subset of $x$ has a lower bound in $x$. Considering $\{\ell,k\}$
        as our subset, suppose we have some lower bound $q \in x$. By definition of meets, $q \leq \ell \land k$, so $\ell \land k \in x$,
        meaning $x \in P_{\angled{\ell \land k}}$. This reasoning also tells
        us that filters on a poset are closed under finite meets.
    \end{proof}
\end{lemma} 
\marginnote{There is an alternate (perhaps more elegant) theorem in MacLane and Moerdijk demonstrating the same property as \cref{lemma:basis}: Let $\calB$ be a basis for a topology on a space $X$. The restriction functor $Sh(X) \to Sh(\calB)$ is an equivalence of categories. They leave the proof as an exercise for the reader.}
\begin{lemma}\label{lemma:basis}
Any sheaf on $P$ is determined by values on the basis $\{P_{\angled{\ell}}\}_{\ell \in \calP}$.
\begin{proof}
    Fix some sheaf $F$ on $P$ and let $O_U$ be open in $P$. There exists some subset $\calQ \subset \calP$ 
    such that $O_U = \bigcup_{\ell \in \calQ}P_{\angled{\ell}}$, so $\{P_{\angled{\ell}}\}_{\ell \in \calQ}$ is a covering of $U$. Since $F$ is a 
    sheaf, it must satisfy
    \[F(O_U) \xlongrightarrow{e} \prod_{\ell \in \calQ} F(P_{\angled{\ell}}) 
    \begin{matrix}
    \xlongrightarrow{p} \\
    \xlongrightarrow{q}
    \end{matrix}
    \prod_{\ell,k \in \calQ} F(P_{\angled{\ell}} \cap P_{\angled{k}}),\]
    which is equal to 
    \[F(O_U) \xlongrightarrow{e} \prod_{\ell \in \calQ} F(P_{\angled{\ell}}) 
    \begin{matrix}
    \xlongrightarrow{p} \\
    \xlongrightarrow{q}
    \end{matrix}
    \prod_{\ell,k \in \calQ} F(P_{\angled{\ell \land k}}).\]
    %Let $s \in F(O_U)$, so $s$ can be written as a family $\{s_\ell\}_{\ell \in \calQ}$ such that $s_\ell \in F(P_{\angled{\ell}})$ for each $\ell \in \calQ$.
    Then to compute behavior of $F$ on $O_U$ such that the sheaf conditions are satisfied, we only need to know how $F$ acts on the basis, since we have rewritten 
    the rest of the diagram in terms of the basis elements.
\end{proof}
\end{lemma}  

Okay, now we are ready to prove \cref{Theorem:P=S}!
\begin{proof}.\newline
    \begin{enumerate}
        \item $Psh(\calP) \to Sh(P)$:  
        \marginnote{We could also be done after this definition, since we have a unique definition of $F$ on the basis, and the fancier theorem says $F(\calB)$ is equivalent to $F(P)$, where $\calB$ is a basis for $P$.}
        Let $E$ be a presheaf on $\calP$. We will use $E$ to define a sheaf $F$ on $P$. We have the natural behavior
        for the basis elements: $F(P_{\angled{\ell}}) := E(\ell)$. 

        As we showed above, all the other open sets follow immediately when we set $F(O_U)$ as the equalizer of the diagram with the covering of choice being the 
        basis elements. Note that this extends to arbitrary coverings, as each open set can be written itself as a union of basis elements, so no matter what we can
        always reduce a covering to its basis elements. 
        
        F maintains all the presheaf requirements as well: suppose $P_{\angled{\ell}} \subset P_{\angled{k}}$. We've shown this implies $\ell \leq k$. 
        Then we have a restriction morphism $E(k) \to E(\ell)$, and by definition this yields a restriction morphism 
        $F(P_{\angled{k}}) \to F(P_{\angled{\ell}})$. Defining $F(O_U)$ to be the equalizer of the given diagram gives us the desired sheaf conditions, as 
        sketched in \cref{Proposition:eq->sheaf}.
        \item $Sh(P) \to Psh(\calP)$:
        Set $E(\ell)$ to $F(P_\ell)$ for all $\ell \in \calP$. 
        
    \end{enumerate}
    
\end{proof}

\newpage

\subsection{Transparency and Sealing Monads}
These transparency and sealing monads are precisely instances of $i_* \circ i^*$, as defined in \cref{ssf}.
\begin{definition}

\label{Definition:t+smonads}
    \begin{enumerate}
        \item[1.] The \textit{transparency monad} $\angled{\ell} \Rightarrow A$ replaces $A$ with whatever part of it can be viewed under policy $\angled{\ell}$. If we define $i:P_{\langle \ell \rangle} \hookrightarrow P,$ the composition $i_* \circ i^*$ defines the unit $\eta_\Rightarrow:Sh(P) \to Sh(P)$ where $A \mapsto (\angled{\ell}\Rightarrow A)$. When the unit is an isomorphism at $A$, we say that $A$ is \textit{$\angled{\ell}$-transparent}.
        
        \item[2.] The \textit{sealing monad} $\angled{\ell} \bullet A$ removes from $A$ whatever part of it can be viewed under policy $\angled{\ell}$. If we define $i:P\setminus P_{\langle \ell \rangle} \hookrightarrow P,$ the composition $i_* \circ i^*$ defines the unit $\eta_\bullet:Sh(P) \to Sh(P)$ where $A \mapsto (\angled{\ell}\bullet A)$.
        \begin{itemize}
            \item The sealing monad can be constructed as the pushout \\ $P_{\angled{\ell}} \sqcup_{P_{\angled{\ell}} \times A} A.$ When the unit is an isomorphism at $A$, we say $A$ is \textit{$\angled{\ell}$-sealed}.
        \end{itemize}
    \end{enumerate}
\end{definition}
\begin{proposition}
\label{Theorem:tfunc}
    Sterling et al. claim the transparency monad "is" the function space $A^{P_{\angled{\ell}}}$, i.e. $A^{P_{\angled{\ell}}} \cong \angled{\ell} \Rightarrow A$. We will sketch a proof for the general case of open $Y$ in $P$.
    \marginnote{You may notice we are treating the open space $P_{\angled{\ell}}$ (and general $Y$) as a sheaf. Recall \cref{Theorem:opens=sheaf}, which proves Sterling et al.'s statement that an open set of $P$ is "the same" as a subterminal sheaf.}
    \begin{proof}(sketch)

        Let $j: Y \hookrightarrow P$ and $A$ be a sheaf on $P$. We want $j_*j^* A \cong A^Y$. One definition of the exponential object on sheaves over $\calO(X)$ is 
        \[A^Y(U) \cong \Hom(Y\mid_U,A\mid_U)\]
        where $Y\mid_U,A\mid_U$ are the sheaves restricted to subsets of $U$. Since $Y\mid_U(V) = 1$ if $V \subseteq U \cap Y$ and $\emptyset$ otherwise, any morphisms (natural transformations) $Y\mid_U \to A\mid_U$ are characterized by $V\subseteq U \cap Y$.  
    
        Let $\alpha$ be a natural transformation and $\alpha_V$ the component associated to $V \subseteq U \cap Y$, i.e. $\alpha_V : Y\mid_U(V) \to A\mid_U(V) = A(V)$. This simplifies to
        $\alpha_V : 1 \to A(V)$. There is a natural isomorphisms between $\Hom(1,A)$ and $A$, for any object $A$, so $\alpha_V$ is naturally isomorphic to $A(V)$. The subsets $V$ form a cover of $U\cap Y$, so we should be able to glue the nontrivial components of $\alpha$ (i.e., $\alpha\mid_{U \cap Y}$) to get $A(U\cap Y)$. Then we have 
        \begin{align*}
            \Hom(Y\mid_U,A\mid_U) &\cong \Hom(Y\mid_{U\cap Y},A\mid_{U\cap Y})\\
            &\cong \{A(V)\mid V \subseteq U \cap Y\}\\
            &\cong A(U \cap Y)
        \end{align*}
        In \cref{ssf}, we demonstrated $j_*j^*A(U) = A(U\cap Y)$ for $j : Y \hookrightarrow P$, so we have $j_*j^*A \cong A^Y$.
    \end{proof}
\end{proposition}



Given space $P$ and arbitrary $\angled{\ell} \in P$, the subspace defining the transparency monad $\angled{\ell}\Rightarrow$ is $P_{\angled{\ell}}$, and the subspace defining the sealing monad $\angled{\ell}\bullet$ is $P_{\bullet \angled{\ell}}$. 

Closed sets are defined to be complements of open sets. It is rare that the complement of an open set will also be open (called clopen sets). Note that $P_{\bullet \angled{\ell}}$ is not guaranteed to be open -- this is where the caveats of \cref{Definition:invsheaf} with respect to the Alexandroff topology come in. That is, for every open $U \in P_{\bullet \angled{\ell}}$, we have $i^*F(U) = F(\uparrow U)$ where $\uparrow U$ is the upward closure of $U$ in P. 
%The sealing monad has a weirder form that is hard to capture accurately with the inclusion adjunction, so we will think of it primarily by the pushout definition.
%\textcolor{red}{tbd. talk to john}
\begin{example}
Let's fix $\calP := \{L \subset M \subset H \subset \top\}$.

    There are only four filters on $\calP$, so we can write all of them out:
    \begin{align*}
        p_0 &= \{L, M, H, \top\} \\
        p_1 &= \{M, H, \top\} \\ 
        p_2 &= \{H, \top\}\\
        p_3 &= \{\top\}
    \end{align*}
    From here, we can also enumerate the principal open sets on $P$:
    \begin{align*}
        P_{\angled{L}} &= \{p_0\}\\
        P_{\angled{M}} &= \{p_0, p_1\}\\
        P_{\angled{H}} &= \{p_0, p_1\, p_2\}\\
        P_{\angled{\top}} &= \{p_0, p_1\, p_2, p_3\}
    \end{align*}


    To get a better idea of how these monads work, we will work through explicitly what $\angled{M} \Rightarrow A$ and $\angled{M} \bullet A$ look like. 
    \begin{enumerate}
        \item $\angled{M} \Rightarrow A$ is the monad defined by composition of the inverse and direct images of $i:P_{\angled{M}}\hookrightarrow P$\\
        \begin{align*}
            &(i_* \circ i^*)A(P_{\angled{L}}) = A(P_{\angled{L}}\cap P_{\angled{M}}) = A(P_{\angled{L}})\\
            &(i_* \circ i^*)A(P_{\angled{M}}) = A(P_{\angled{M}}\cap P_{\angled{M}}) = A(P_{\angled{M}})\\
            &(i_* \circ i^*)A(P_{\angled{H}}) = A(P_{\angled{H}}\cap P_{\angled{M}}) = A(P_{\angled{M}})\\
            &(i_* \circ i^*)A(P_{\angled{\top}}) = A(P_{\angled{\top}}\cap P_{\angled{M}}) = A(P_{\angled{M}})
        \end{align*}
        \item $\angled{M} \bullet A$ is the monad defined by composition of the inverse and direct images of $i: P_{\bullet \angled{M}} \hookrightarrow P$
            \begin{align*}
                &(i_* \circ i^*)A(P_{\angled{L}}) = A(\uparrow(P_{\angled{L}} \cap P_{\bullet \angled{\ell}})) = A(\emptyset) = 1\\
                &(i_* \circ i^*)A(P_{\angled{M}}) = A(\uparrow(P_{\angled{M}} \cap P_{\bullet \angled{\ell}})) = A(\emptyset) = 1\\
                &(i_* \circ i^*)A(P_{\angled{H}}) = A(\uparrow(P_{\angled{H}} \cap P_{\bullet \angled{\ell}})) = A(\uparrow(\{p_2\})) = A(P_{\angled{H}})\\
                &(i_* \circ i^*)A(P_{\angled{\top}}) = A(\uparrow(P_{\angled{\top}} \cap P_{\bullet \angled{\ell}})) = A(\uparrow(\{p_2,p_3\})) = A(P_{\angled{\top}})
            \end{align*}
        This aligns with intuition that sealing at level $\angled{\ell}$ preserves all data "higher" than $\ell$, but collapes/makes all data at or "lower" than $\ell$ indistinguishable. 
    \end{enumerate}
\begin{exercise}
\label{ex3e}
\sloppy
    The authors mention that the sealing monad also has a pushout construction. Check that evaluating $P_{\angled{M}} \sqcup_{P_{\angled{M}} \times A} A$ at the principal sets ($P_{\angled{\ell}}$ for $\ell \in \calP$) agrees with the composition of the direct and inverse images of the inclusion map. 

    (Hint : use \cref{Example:fxg}; don't worry about sheafification)
    See Solution \ref{ex3s}
\end{exercise}
\end{example}

\begin{proposition}
\label{Proposition:t->s->1}
The $\angled{\ell}$-transparent part of an $\angled{\ell}$-sealed sheaf is trivial, i.e. $(\angled{\ell} \Rightarrow (\angled{\ell} \bullet A))\cong \{*\}$. 
\begin{proof}
    Let's expand the definition of $\angled{\ell} \Rightarrow$ in the above expression:
    \[i_*(i^*(\angled{\ell} \bullet A))(V) = (i^*(\angled{\ell} \bullet A))(i^{-1} V) = (i^*(\angled{\ell} \bullet A))(V \cap P_{\angled{\ell}}) = (\angled{\ell} \bullet A)(V \cap P_{\angled{\ell}}) \]
    We know $V \cap P_{\angled{\ell}} \subset P_{\angled{\ell}}$, so from our computation above we can see $(\angled{\ell} \bullet A)(V \cap P_{\angled{\ell}}) = \{*\}$ for any $V \in P$.
\end{proof}
\end{proposition}

%\begin{proposition}
   % Any sheaf $A \in Sh(P)$ can be reconstructed as the pullback $(\angled{\ell} \Rightarrow A) \times_{\angled{\ell} \bullet (\angled{\ell} \Rightarrow A)} \angled{\ell} \bullet A$. 
%\end{proposition}

\subsection{Noninterference}
We've made it! \\The noninterference property will be represented by the constant definition. That is, if we have a morphism from higher security to lower security, this morphism satisfies noninterference if it is constant. 

Recall that we aim to view these sheaves as types: we can think of $A(U)$ where $U \subset P$ as the data of type $A$ which has some associated security restrictions/permissions. If we say a morphism $A \to B$ is constant, this means that the behavior of $f$ does not reveal any information about data of type $A$. 



\begin{theorem}
    Any map $\angled{\ell} \bullet A \to \texttt{bool}$ is constant.
    \begin{proof}
         The boolean sheaf \texttt{bool} assigns to each open the set of boolean values $\{\texttt{true},\texttt{false}\}$. 

        Let's see what happens when we apply $\angled{\ell} \Rightarrow \texttt{bool}$. This evaluates to \texttt{bool}$(P_{\angled{\ell}} V) = \{\texttt{true},\texttt{false}\}$ for any open $V \in P$. By function extensionality, $\angled{\ell} \Rightarrow \texttt{bool} = \texttt{bool}$ for all $\ell$, so \texttt{bool} is $\angled{\ell}$-transparent for all $\ell$, meaning $\angled{\ell}\Rightarrow\texttt{bool}\cong\texttt{bool}$. 

        The unit $\eta_{\Rightarrow}: Id \to (\angled{\ell} \Rightarrow \_)$ is a natural transformation, so the following diagram commutes:

        \[\begin{tikzcd}[row sep=large, column sep=large]
        \angled{\ell} \bullet A \arrow[r, "f"] \arrow[d, "\eta_{\Rightarrow}"'] & \texttt{bool} \arrow[d, "\eta_{\Rightarrow}"] \\
        \angled{\ell} \Rightarrow (\angled{\ell} \bullet A) \arrow[r,"g"  '] & \angled{\ell} \Rightarrow \texttt{bool} \arrow[ul, phantom, very near start]
        \end{tikzcd} \cong 
        \begin{tikzcd}[row sep=large, column sep=large]
        \angled{\ell} \bullet A \arrow[r, "f"] \arrow[d, "\eta_{\Rightarrow}"'] & \texttt{bool} \arrow[d, "\eta_{\Rightarrow}"] \\
        \{*\} \arrow[r,  "g"'] & \texttt{bool} \arrow[ul, phantom, very near start]
        \end{tikzcd}
        \]
        The diagram on the left comes from \cref{Proposition:t->s->1} and the fact that \texttt{bool} is $\angled{\ell}$-transparent for all $\ell$. This simplifies to 
        \[
        \begin{tikzcd}
        \angled{\ell} \bullet A \arrow[r, "f"] \arrow[dr, "!"] & \texttt{bool} \\
                                & \{*\} \arrow[u,"g"]
        \end{tikzcd}
        \]
        Since $f$ factors through the terminal object, $f$ must be constant.
    \end{proof}
\end{theorem}
 
Since \texttt{bool} is a constant sheaf, it is essentially public. So, the above theorem is stating the mapping to bool from any type of $\angled{\ell}$ sealed data will not reveal anything about this "more secure" data. 

%A function $f: A \to B$ is \textit{weakly constant} if, for any $x,y \in A$ we have $f(x) = f(y)$. 

\begin{theorem}
    Morphisms $\angled{\ell} \bullet A \to B$ are in bijective correspondence with morphisms $A \to B$ that restricts to a constant function under $\angled{\ell}$.
    \begin{proof}
        We take "restricts under $\angled{\ell}$" to mean "under application of the $\angled{\ell}$ transparency monad". Recall $\eta_\bullet$ refers to the unit for sealing.
        \begin{enumerate}
            \item Begin with $f : \angled{\ell} \bullet A \to B$.

            We can use the pushout definition of the sealing monad to get the following diagram:
            \[\begin{tikzcd}[row sep=large, column sep=large]
                P_{\angled{\ell}} \times A \arrow[r] \arrow[d] & A \arrow[d, "\eta_\bullet"] \arrow[dr, "f'"]\\
                P_{\angled{\ell}} \arrow[r] & \angled{\ell} \bullet A \arrow[r, "f"] & B
            \end{tikzcd}\]
            where $f'$ is defined to be $f \circ \eta_\bullet$. Based on our understanding of restriction under $\angled{\ell}$, we want to see how $\angled{\ell}\Rightarrow f'$, mapping $\angled{\ell} \Rightarrow A \to \angled{\ell} \Rightarrow  B$ behaves, so really we only care about 
            \[\begin{tikzcd}
                A \arrow[r, "\eta_\bullet"] \arrow[rr, bend left=45, "f'"] & \angled{\ell} \bullet A \arrow[r,"f"] & B
            \end{tikzcd}\]
            
            We now use the functoriality of $\angled{\ell}\Rightarrow$ to have that the following diagram commutes,

            \[
            \begin{tikzcd}
                \angled{\ell} \Rightarrow A \arrow[r,"\angled{\ell}\Rightarrow \eta_\bullet"]\arrow[rr,bend left=45, "\angled{\ell}\Rightarrow f'"] & \angled{\ell} \Rightarrow (\angled{\ell} \bullet A) \arrow[r,"\angled{\ell}\Rightarrow f"] & \angled{\ell} \Rightarrow B
            \end{tikzcd}
            \] 
            which simplifies to 
            \[
            \begin{tikzcd}
                \angled{\ell} \Rightarrow A \arrow[r,"!"]\arrow[rr,bend left=45, "\angled{\ell}\Rightarrow f'"] & \{*\} \arrow[r,"\angled{\ell}\Rightarrow f"] & \angled{\ell} \Rightarrow B
            \end{tikzcd}
            \] 
            So, since $\angled{\ell}\Rightarrow f'$ factors through the terminal object, it must be constant, meaning the $f'$ defined by $f \circ \eta_\bullet$ restricts to a constant function under $\angled{\ell}$.
            \item Begin with an $f' : A \to B$ such that $\angled{\ell} \Rightarrow f'$ is constant. 
            
            We again want to observe the pushout diagram for $\angled{\ell} \bullet A$:
            \[
            \begin{tikzcd}
                P_{\angled{\ell}} \times A \arrow[r] \arrow[d] & A \arrow[d, "\eta_\bullet"] \arrow[dr, "f'"]\\
                P_{\angled{\ell}} \arrow[r] & \angled{\ell} \bullet A & B
            \end{tikzcd}
            \]
            Note we do not have a defined arrow $\angled{\ell} \bullet A$ to $B$. By the universal property of the pushout, if we have morphisms $A \to B$ and $P_{\angled{\ell}} \to B$, then there exists a unique morphism $\angled{\ell} \bullet A \to B$ such that everything is nice etcetera. To obtain a morphism $P_{\angled{\ell}} \to B$, we use the constant under $\angled{\ell}$ assumption for $f'$. 

            $\angled{\ell} \Rightarrow f'$ constant means for any $x\in \angled{\ell} \Rightarrow A$, we have $(\angled{\ell} \Rightarrow f')(x) = c$ for some constant $c \in \angled{\ell}\Rightarrow B$. \cref{Theorem:tfunc} states that the transparency monad is isomorphic to the function space $A^{P_{\angled{\ell}}}$ (with open sets being treated as subterminal sheaves), so this constant $c$ is isomorphic to a map $P_{\angled{\ell}} \to B$, so we can complete our diagram to get 
            \[
            \begin{tikzcd}
                P_{\angled{\ell}} \times A \arrow[r] \arrow[d] & A \arrow[d, "\eta_\bullet"]\arrow[ddr, bend left, "f'"] & \\
                P_{\angled{\ell}} \arrow[r] \arrow[drr, bend right, "c"] & \angled{\ell} \bullet A \arrow[dr, dashed, "!f"] & \\
                & & B
            \end{tikzcd}
            \]
            For each $f' : A \to B$ restricting to a constant function under $\angled{\ell}$, there exists a unique $f : \angled{\ell} \bullet A \to B$.
        \end{enumerate}  
    \end{proof}
\end{theorem}
The noninterference intuition for the above theorem is that, if we have a map which relies on sealed (secret,private,etc) data, then when we try to take a "less secure" view, the behavior of this map is indistinguishable.
%balloon diagrams to make sense of all this (or maybe not):
\section{Solutions}

\begin{solution}
\label{ex1s}
Solution to Exercise \ref{ex1e}:

Let $U$ be open in $Y$. Then $U$ is an upper set, meaning if $a \in U$ and $a \leq b$, then $b \in U$. Suppose we have $x \in f^{-1}(U)$ and $y\in X$ such that $x \leq y$. We wish to show $y \in f^{-1}(U)$. 

Since $x \in f^{-1}(U)$, we know $f(x) \in U$. By monotonicity, we know $f(x) \leq f(y)$, meaning $f(y) \in U$. Then $y \in f^{-1}(U)$, so $f^{-1}(U)$ is an upper set and thus open. 

\end{solution}

%Let's check these are open sets:
\begin{solution}
\label{ex2s}
Solution to Exercise \ref{ex2e}:
    \begin{enumerate}
        \item The empty set, along with being an upper set, is also a lower set (for the same vacuous reason). Then we can write the set $\{x \mid x \cap \emptyset \neq \emptyset\} = \emptyset$, so the empty set fits this characterization. Now we look at $\{ x \mid x\cap P \neq \emptyset\},$ but this in fact contains every $x$ because each $x$ is nonempty, so this set is equal to $P$.
        \item Arbitrary unions: we want to write $\bigcup_{i \in I} \{x \mid U_i\vdash x\}$ as \newline $\{x \mid\; ? \vdash x\}$.What does it mean for some $x'$ to be in this union? That there is some $U_j$ such that $x' \cap U_j \neq \emptyset$, which means there exists some $y$ such that $y \in x'$ and $y \in U_j$. So we can reformulate this as saying $x$ is in the arbitrary union if there exists some $j \in I$ and $y \in x$ such that $y \in U_j$. The existence of some $j$ can be rephrased as requiring there exist some $y\in x$ such that $y \in \bigcup_{i \in I} U_i$. Then this is equivalent to the existence of some\newline  $y \in x \cap \bigcup_{i \in I}U_i$, so we can rewrite the arbitrary union as \[\{x \mid x \cap \bigcup_{i \in I}U_i\}.\] The arbitrary union of lower sets is also a lower set --- the proof is very similar to that for upper sets, you can do it if you are not convinced. So, we have that these sets are closed under arbitrary unions. 
        \item Finite intersections: Similarly, we can rewrite $O_V \cap O_U$ as \newline $\{x \mid (U \cap V)\vdash x\}$, and lower sets are also closed under finite intersections.
    \end{enumerate}
\end{solution}

\begin{solution}
\label{ex3s}
    Solution to Exercise \ref{ex3e}:

    We will compute $\angled{M} \bullet A$ as the pushout of $P_{\angled{M}} \times A$. We will use \cref{Example:fxg}, which was for presheaves. Because this space $Sh(P)$ is particularly nice, we don't have to do any fancy sheafifying after computing the pointwise values.
        Recall $P_{\angled{M}}$ is being treated as a subterminal sheaf, so since $P_{\angled{H}} \not\subset P_{\angled{M}}$ for example,
        $P_{\angled{M}}$ applied to $P_{\angled{H}}$ will be $\emptyset$. On the other hand, both $P_{\angled{M}}$ and $P_{\angled{L}}$ subset $P_{\angled{M}}$, so the application returns 1.\\
        These are the resulting values of applying the monad:
        \begin{align*}
            &(\angled{M} \bullet A)(P_{\angled{L}}) = 1\\
            &(\angled{M} \bullet A)(P_{\angled{M}}) = 1\\
            &(\angled{M} \bullet A)(P_{\angled{H}}) = A(P_{\angled{H}})\\
            &(\angled{M} \bullet A)(P_{\angled{\top}}) = A(P_{\angled{\top}})
        \end{align*}  
\end{solution}

\begin{solution}
\label{eqmons}
Solution to Lemma \ref{eqmonl} (sketch)

Let $e$ be an equalizer map for $A \overset{p}{\underset{q}{\rightrightarrows}} B$. We want to show $e$ is a monomorphism, meaning for any $f:C \to A, g:C\to A$, if we have $e \circ f = e \circ g$, this implies $f = g$. So, suppose $e \circ f = e \circ g$.

By equalizerness, we know $p \circ e = q \circ e$, so we can set $p \circ e \circ f = q \circ e \circ f$. Then $e \circ f$ is an equalizer of $p$ and $q$. By the universal property, there exists a unique $u$ such that $e \circ u = e \circ f$, and uniqueness of $u$ gives us $u = f = g$, so we have $f = g$.

The coequalizer $\implies$ epimorphism reasoning is very similar.
\end{solution}
%Fix $A \in Sh(P)$, and recall $\calP = \{L \subset M \subset H \subset \top\}$. For our example we will work with $\langle M \rangle$. 

